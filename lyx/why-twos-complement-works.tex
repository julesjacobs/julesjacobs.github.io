%% LyX 2.3.2-2 created this file.  For more info, see http://www.lyx.org/.
%% Do not edit unless you really know what you are doing.
\documentclass[english]{article}
\usepackage[T1]{fontenc}
\usepackage[latin9]{inputenc}
\usepackage{amsmath}
\usepackage{amssymb}

\makeatletter

%%%%%%%%%%%%%%%%%%%%%%%%%%%%%% LyX specific LaTeX commands.
%% Because html converters don't know tabularnewline
\providecommand{\tabularnewline}{\\}

\makeatother

\usepackage{babel}
\begin{document}
\title{Why two's complement works}

\maketitle
There are unsigned integers and signed integers. The first bit in
a signed integer indicates whether it is positive or negative. You'd
expect that CPUs have different instructions to do arithmetic with
signed integers and unsigned integers. That's partly true: x86 has
DIV to divide unsigned integers and IDIV to divide signed integers,
but there is only one ADD instruction, only one SUB instruction, and
only one MUL instruction. These instructions work for both signed
and unsigned integers. How can that be?!

This is the magic of two's complement. The bit patterns of signed
integers are precisely such that ordinary unsigned arithmetic gives
the correct sign bit. For four bit numbers the representation is as
follows:
\begin{center}
\begin{tabular}{|c|c|c|}
\hline 
bit pattern & unsigned value & signed value\tabularnewline
\hline 
\hline 
0000 & 0 & 0\tabularnewline
\hline 
0001 & 1 & 1\tabularnewline
\hline 
0010 & 2 & 2\tabularnewline
\hline 
0011 & 3 & 3\tabularnewline
\hline 
0100 & 4 & 4\tabularnewline
\hline 
0101 & 5 & 5\tabularnewline
\hline 
0110 & 6 & 6\tabularnewline
\hline 
0111 & 7 & 7\tabularnewline
\hline 
1000 & 8 & -8\tabularnewline
\hline 
1001 & 9 & -7\tabularnewline
\hline 
1010 & 10 & -6\tabularnewline
\hline 
1011 & 11 & -5\tabularnewline
\hline 
1100 & 12 & -4\tabularnewline
\hline 
1101 & 13 & -3\tabularnewline
\hline 
1110 & 14 & -2\tabularnewline
\hline 
1111 & 15 & -1\tabularnewline
\hline 
\end{tabular}
\par\end{center}

The operation +1 goes forward by one step in this table, and wraps
around from the end to the start. The operation -1 goes in reverse.
If we start with the value 0000 and subtract 1, we end up with 1111.
For signed values that bit pattern indeed represents -1. If we keep
subtracing -1's, it goes back to -8, which is the minimum representable
signed integer, and then wraps around to +7. You see that the operations
+1 and -1 work exactly the same on a bit pattern, regardless of whether
it represents a signed or unsigned value. The only difference is the
meaning of the bit patterns: 1111 represents 15 if it's an unsigned
value, and -1 if it's a signed value. So if we \emph{print} an integer,
we need to know whether it's signed or unsigned, but as long as we
only do +1 and -1 we don't need to know whether it's signed or unsigned.

Adding bigger amounts, say, +3, is independent of whether it's signed
or unsigned too, because adding +5 is the same as +1+1+1, so if adding
+1 is independent of whether it's signed or unsigned, then +3 is too.
Try an example: if we start with 1001 and add +2 we end up with 1011.
As an unsigned value, that was 9+2=11, and with a signed value, that
was -7+2=-5. Both correct! Similarly, subtracting bigger amounts,
or adding negative amounts, also works. Multiplication be expressed
as repeated addition, so that's independent too. This does \emph{not}
work for division, because division cannot be expressed as repeated
addition of +1. That's why x86 has separate DIV and IDIV instructions

Technically, that argument only shows that addition $a+b$, subtraction
$a-b$, and multiplication $a\cdot b$ is independent of whether $a$
is signed or unsigned. You'll probably not be suprised that it's also
independent of whether $b$ is signed or unsigned. If you're familiar
with modular arithmetic, this is because $(a\mod16)+(b\mod16)=(a+b)\mod16$
and similarly for subtraction and multiplication. The only difference
between signed and unsigned is which representative from $\mathbb{Z}$
we assign to each equivalence class in $\mathbb{Z}/16\mathbb{Z}$.
That's just a fancy way of saying that when we print a bit pattern,
what we print depends on whether it represents a signed or unsigned
value.
\end{document}

%% LyX 2.3.2-2 created this file.  For more info, see http://www.lyx.org/.
%% Do not edit unless you really know what you are doing.
\documentclass[english]{article}
\usepackage[T1]{fontenc}
\usepackage[latin9]{inputenc}
\usepackage{babel}
\usepackage[unicode=true]
 {hyperref}
\usepackage{listings}
\renewcommand{\lstlistingname}{Listing}

\begin{document}
\title{Leapfrog and Verlet are the same method}

\maketitle
Leapfrog and Verlet are two popular methods to integrate Newton's
equations of motion in physics simulations and games. These methods
occupy a sweet spot between Euler's method (the simplest method) and
higher order methods. They are almost as simple as Euler's method
and use only one force calculation per time step, yet they have crucial
advantages: they are second order accurate (compared Euler's method
which is of order one), and they are symplectic and time reversible.
Because they have such similar properties I wondered if Leapfrog and
Verlet are actually the same method -- after all, how many second
order sympletic methods can there be? It turns out that they are indeed
equivalent, a fact apparently well known among numerical methods people.
Their equivalence is easier to understand with code than with math
notation, so that's what I'll do here.

I'm assuming that we have a function $a(x)$ to compute the forces/accelerations
from the positions, and $x$ and $v$ are initialised to the initial
positions and velocities. Here is Verlet:

\begin{lstlisting}
for(i = 0..n){
  x_prev = x
  x += v*dt + a(x)*dt/2
  v += (a(x_prev) + a(x))*dt/2
}
\end{lstlisting}
And here is Leapfrog:

\begin{lstlisting}
for(i = 0..n){
  x += v*dt
  v += a(x)*dt
}
\end{lstlisting}
At first sight there's no way that they could be equivalent, because
Verlet computes $x(t)$ and $v(t)$ at multiples of the time step
$t=0,1,2,\dots$, whereas Leapfrog computes $x(t)$ at $t=0,1,2,\dots$
and $v(t)$ at $t=0+\frac{1}{2},1+\frac{1}{2},2+\frac{1}{2},\dots$
shifted one half step from each other (that's why it's called Leapfrog:
the $x$ and $v$ values leapfrog over each other). However, at the
start of the simulation we're given $x(0)$ and $v(0)$, whereas Leapfrog
requires $x(0)$ and $v(\frac{1}{2})$. To get Leapfrog started we
must compute $v(\frac{1}{2})$ first, which we can do with one step
of Euler's method with $\Delta t=\frac{1}{2}$. Similarly, at the
end we've got $x(n)$ and $v(n+\frac{1}{2})$ whereas we'd like to
know $v(n)$, which we can do by doing one step of Euler's method
backward with $\Delta t=-\frac{1}{2}$. That gives us the corrected
Leapfrog:

\begin{lstlisting}
v += a(x)*dt/2
for(i = 0..n){
  x += v*dt
  v += a(x)*dt
}
v -= a(x)*dt/2
\end{lstlisting}
Let's rewrite this in an apparently silly way by splitting the $v$
update in half:

\begin{lstlisting}
v += a(x)*dt/2
for(i = 0..n){
  x += v*dt
  v += a(x)*dt/2
  v += a(x)*dt/2
}
v -= a(x)*dt/2
\end{lstlisting}
Think about what this is doing by unrolling this loop in your mind:

\begin{lstlisting}
v += a(x)*dt/2
x += v*dt
v += a(x)*dt/2
v += a(x)*dt/2
x += v*dt
v += a(x)*dt/2
v += a(x)*dt/2
... n times ...
x += v*dt
v += a(x)*dt/2
v += a(x)*dt/2
v -= a(x)*dt/2
\end{lstlisting}
The last two updates cancel each other out, so we can remove both.
A different way of looking at it emerges:

\begin{lstlisting}
v += a(x)*dt/2
x += v*dt
v += a(x)*dt/2

v += a(x)*dt/2
x += v*dt
v += a(x)*dt/2

... n times ...

v += a(x)*dt/2
x += v*dt
v += a(x)*dt/2
\end{lstlisting}
In other words, the Leapfrog code is equivalent to this:

\begin{lstlisting}
for(i = 0..n){
  v += a(x)*dt/2
  x += v*dt
  v += a(x)*dt/2
}
\end{lstlisting}
An iteration of this loop is exactly the same as an iteration of Verlet!
Can you see why? (Hint: incorporate the first $v$ update into the
subsequent $x$ and $v$ updates).

\subsection*{Second variant}

Instead of advancing $v$ by half a timestep at the start and end
of Leapfrog, we could also advance $x$ by half a timestep to get
the second variant of Leapfrog:

\begin{lstlisting}
x += v*dt/2
for(i=0 to n){
  v += a(x)*dt
  x += v*dt
}
x -= v*dt/2
\end{lstlisting}
We can do the same rewrite and move everything into the loop:

\begin{lstlisting}
for(i = 0..n){
  x += v*dt/2
  v += a(x)*dt
  x += v*dt/2
}
\end{lstlisting}
This variant has the advantage that $a(x)$ is only computed once
per iteration. We don't \emph{really }need to compute $a(x)$ twice
in the previous variant of Leapfrog-Verlet either, because the second
call to $a(x)$ will be the same as the first call in the subsequent
iteration, so we could save that instead of recomputing it. That complicates
the code a bit, so I find this second variant nicer. By incorporating
the first $x$ update into the subsequent $v$ and $x$ updates we
obtain the second variant of Verlet:

\begin{lstlisting}
for(i = 0..n){  
  v_prev = v
  v += a(x+v*dt/2)*dt
  x += (v_prev + v)*dt/2
}
\end{lstlisting}
This variant of Verlet also has the advantage of only computing $a(x)$
once. For some reason the other variant seems to be \href{https://en.wikipedia.org/wiki/Verlet_integration\#Velocity_Verlet}{more popular}.

\subsection*{Conclusion}

In my opinion, the best way to write Leapfrog-Verlet is this:

\begin{lstlisting}
for(i = 0..n){
  x += v*dt/2
  v += a(x)*dt
  x += v*dt/2
}
\end{lstlisting}
The advantage is that it's pretty, computes both $x$ and $v$ at
$t=0,1,2,\dots$, and doesn't use any state other than $(x,v)$. The
disadvantage is that it updates $x$ twice per iteration, instead
of once as Leapfrog does. This is likely to be of negligible cost
compared to computing $a(x)$, but if you really care about it then
use the second variant of Leapfrog. Just keep in mind Leapfrog computes
$x$ at shifted time steps, so if you use $\frac{1}{2}mv^{2}+U(x)$
to compute the energy you'll get an incorrect value.
\end{document}

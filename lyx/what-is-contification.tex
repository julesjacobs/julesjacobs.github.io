%% LyX 2.3.2-2 created this file.  For more info, see http://www.lyx.org/.
%% Do not edit unless you really know what you are doing.
\documentclass[english]{article}
\usepackage[T1]{fontenc}
\usepackage[latin9]{inputenc}
\usepackage{babel}
\usepackage[unicode=true]
 {hyperref}
\usepackage{listings}
\renewcommand{\lstlistingname}{Listing}

\begin{document}
\title{What is contification?}

\maketitle
Inlining is one of the most important compiler optimisations. The
danger with inlining is that it can cause code size explosion, because
the function that is inlined is copied to multiple call sites. However,
if the function is only called in one place, then code duplication
doesn't occur, because the original function can be deleted after
inlining. Another way to look at this transformation is that we remove
the call instruction and replace it with a jump to the entry point
of the function, and we replace the return instruction inside the
function with a jump back to the point where the function is called.

Contification is based on the clever observation that you can do this
transformation whenever a function returns to only one location, even
if the function is called in multiple places. Here's the simplest
example:

\begin{lstlisting}
function f(y){
  return ...
}

if(cond){   
  x = f(E1) 
}else{   
  x = f(E2) 
}

<REST ...>
\end{lstlisting}

The function f is called in two places, but the function returns to
the same place. The compiler can therefore replace each call to f
with a jump to f's entry point, and replace the return inside f with
a jump to the code after the if-else:

\begin{lstlisting}
label f:    
  x = ...
  goto REST

if(cond){    
  y = E1    
  goto f  
}else{    
  y = E2    
  goto f  
}

label REST: <...>
\end{lstlisting}

This eliminates the function call overhead, but more importantly,
it allows other compiler optimisations to kick in.

Analysing whether a function returns to only one place is a little
bit tricky. The most powerful contification analyses can reason about
tail calls. This allows them to optimise tail recursive functions
that are only called in one place:

\begin{lstlisting}
function loop(x, n){
  if(n==0) return ...
  else return loop(g(x), n-1)
}

result = loop(a, k)

<REST ...>
\end{lstlisting}

We actually have two calls to loop, one from the outside and one tail
recursive call from loop itself. Intuitively you'll understand that
this pattern is equivalent to an actual loop, and contification can
turn this code into an actual loop:

\begin{lstlisting}
label loop:
  if(n==0){
    result = ...
    goto REST
  }else{
    x = g(x)
    n = n-1
    goto loop
  }

label REST: <...>
\end{lstlisting}

It can do this because the contification analysis determines that
loop transitively (through tail calls) always returns to the same
point. Note that we could have eliminated the two calls to f in the
previous example by inlining f twice. Simple inlining truly does not
work for the tail recursive example; no amount of inlining will eliminate
the recursive call. Contification doesn't work when there are multiple
external calls to loop, so contification is not more general than
inlining, and inlining is not more general than contification. Perhaps
one can come up with an optimisation that generalises both: a version
of inlining that doens't inline a function a single function call,
but inlines multiple calls with a single return point simultaneously.
In other words, a version of inlining that inlines a function \emph{return}
rather than a function \emph{call}.

In the general case we have multiple functions, multiple tail calls,
and multiple non-tail calls. The contification analysis tries to find
out, for each function, if there exists a point in the program that
the function always returns to, possibly through zero or more tail
calls. If you're interested in the details, read the paper \href{https://www.cs.purdue.edu/homes/suresh/502-Fall2008/papers/contification.pdf}{Contification Using Dominators}
by Matthew Fluet and Stephen Weeks. They work with a functional style
IR with continutations. In this post I've used an imperative IR, which
I hope makes the contification transformation a little bit easier
to understand: it's just replacing call and return instructions with
jumps. I don't think that the idea of contification has fully penetrated
conventional compilers, but maybe it should.
\end{document}

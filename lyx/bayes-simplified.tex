%% LyX 2.3.2-2 created this file.  For more info, see http://www.lyx.org/.
%% Do not edit unless you really know what you are doing.
\documentclass[english]{article}
\usepackage[T1]{fontenc}
\usepackage[latin9]{inputenc}
\usepackage{amsmath}
\usepackage{babel}
\begin{document}
\title{Bayes' rule simply}

\maketitle
Bayes' rule is usually written 
\begin{align*}
P(\theta|x) & =P(x|\theta)\frac{P(\theta)}{P(x)}
\end{align*}

In practice we're trying to learn about some model parameter $\theta$
given some observation $x$. The model $P(x|\theta)$ tells us how
observations are influenced by the model parameter. This seems simple
enough, but a small change in notation reveals how simple Bayes' rule
is. Let us call $P(\theta)$ the prior on $\theta$ and $P'(\theta)$
the posterior on theta. Then Bayes' rule says:

\begin{align*}
P'(\theta) & \propto P(x|\theta)P(\theta)
\end{align*}
We got rid of the denominator $P(x)$ because it's just a normalisation
to make the total probability sum to 1, and say that $P'(\theta)$
is proportional to $P(x|\theta)P(\theta)$. The value $P(x|\theta)P(\theta)=P(x,\theta)$
is the joint probability of seeing a given pair $(x,\theta)$, so
we can also write Bayes' rule as:

\begin{align*}
P'(\theta)\propto & P(x,\theta)
\end{align*}
So up to normalisation, the posterior is just substituting the actual
observation $X=x$ into the joint distribution. How can we interpret
this? Imagine that we have a robot whose current state of belief is
given by $P(x,\theta)$ and that $x,\theta$ only have a finite number
of possible values, so that the robot has stored a finite number of
probabilities $P(x,\theta)$, one for each pair $(x,\theta)$. Suppose
that the robot now learns $X=x$ by observation. What does it do to
compute its posterior belief? It first sets $P(y,\theta)=0$ for all
$y\neq x$ because the actual observed value is $x$. Then it renormalises
the probabilities to make $P(x,\theta)$ sum to 1 again. That's all
Bayes' rule is: simply delete the possibilities that are incompatible
with the observation, and renormalise the remainder.
\end{document}

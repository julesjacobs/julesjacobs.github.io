%% LyX 2.3.2-2 created this file.  For more info, see http://www.lyx.org/.
%% Do not edit unless you really know what you are doing.
\documentclass[english]{article}
\usepackage[T1]{fontenc}
\usepackage[latin9]{inputenc}
\usepackage{babel}
\usepackage{amsmath}
\usepackage[unicode=true]
 {hyperref}
\begin{document}
\title{Proof that the calculus sin and cos functions equal the geometric
sin and cos}

\maketitle
The trigonometric functions can be defined in many different ways:
as a power series, as a differential equation, using the complex exponential,
and so on. That those definitions are equivalent is usually shown
in a real analysis course. However, students first learn sin and cos
in the context of geometry. The geometric sin and cos can also be
defined in many ways, e.g. in terms of angles and side lenghts of
triangles, or in terms of a point moving around the unit circle. It
is again easy to show that those geometric definitions are equivalent.
What's sometimes missing is a proof that the calculus definitions
are equivalent to the geometric definitions. \href{https://www.khanacademy.org/math/ap-calculus-ab/ab-differentiation-1-new/ab-2-7/a/proving-the-derivatives-of-sinx-and-cosx}{Khan academy has a nice proof using areas.}
That argument is a little bit technical and it is difficult to find
that proof. In this post I'll give an argument using lengths, that
is in my opinion a bit easier to come up with and to remember. It
is perhaps a bit less ironclad than Khan's proof, because it requires
you to accept facts about arclengths of curves, whereas Khan's proof
only involves comparing areas.

Here we go. We define $\sin$ and $\cos$ as the functions satisfying
the differential equation
\begin{align*}
\cos' & =\sin\\
\sin' & =-\cos
\end{align*}
with initial conditions $\cos(0)=1$ and $\sin(0)=0$. We show that
the point $(\cos(t),\sin(t))$ travels around the unit circle at unit
speed. To show that the point stays on the unit circle we must show
$\cos(t)^{2}+\sin(t)^{2}=1$. At $t=0$ the equation holds due to
the initial conditions. We differentiate 
\begin{align*}
(\cos(t)^{2}+\sin(t)^{2})' & =2\cos(t)\cos'(t)+2\sin(t)\sin'(t)\\
 & =2\cos(t)\sin(t)-2\sin(t)\cos(t)\\
 & =0
\end{align*}
So the value of $\cos(t)^{2}+\sin(t)^{2}$ doesn't change, i.e. it
stays equal to $1$. We also need to show that the point $(\cos(t),\sin(t))$
moves around the unit circle at unit speed. After all, there are many
functions $f,g$ such that the point $(f(t),g(t))$ stays on the unit
circle that are not equal to $\cos$ and $\sin$. The speed of the
point is 
\begin{align*}
\cos'(t)^{2}+\sin'(t)^{2} & =\sin(t)^{2}+(-\cos(t))^{2}\\
 & =\sin(t)^{2}+\cos(t)^{2}\\
 & =1
\end{align*}
The point $(\cos(t),\sin(t))$ indeed moves around the unit circle
at unit speed. This implies that $(\cos(t),\sin(t))$ give the coordinates
of the point on the unit circle if we take an angle $t$ measured
in radians from the $x$-axis, because that's how radians are defined.
Therefore the calculus $\sin$ and $\cos$ indeed agree with the geometric
$\sin$ and $\cos$.
\end{document}

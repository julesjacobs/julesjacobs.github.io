\documentclass[a4paper, 11pt]{article}
\usepackage[a4paper,left=2.5cm,right=2.5cm,top=2.5cm,bottom=2.5cm]{geometry}
\usepackage[utf8]{inputenc}
\usepackage[T1]{fontenc}
\usepackage[bitstream-charter]{mathdesign}
\usepackage{microtype}
\usepackage{textcomp}

\usepackage{mathtools}
\newcommand\SetSymbol[1][]{\nonscript\:#1\vert\allowbreak\nonscript\:\mathopen{}}
\providecommand\given{} % to make it exist
\DeclarePairedDelimiterX\Set[1]\{\}{\renewcommand\given{\SetSymbol[\delimsize]}#1}

\usepackage{hyperref}
\hypersetup{
    colorlinks=true,
    linkcolor=blue,
    filecolor=magenta,
    urlcolor=blue,
}
\usepackage{listings}
\usepackage{color}
\definecolor{lightgray}{rgb}{.9,.9,.9}
\definecolor{darkgray}{rgb}{.4,.4,.4}
\definecolor{purple}{rgb}{0.65, 0.12, 0.82}

\lstdefinelanguage{JavaScript}{
  keywords={typeof, new, true, false, catch, function, return, null, catch, switch, var, in, while, do, else, case, break, val, then, Inductive, Fixpoint, match, end},
  keywordstyle=\color{blue},
  ndkeywords={class, export, boolean, throw, implements, import, this},
  ndkeywordstyle=\color{darkgray}\bfseries,
  identifierstyle=\color{black},
  sensitive=false,
  comment=[l]{//},
  morecomment=[s]{/*}{*/},
  commentstyle=\color{purple}\ttfamily,
  stringstyle=\color{red}\ttfamily
}

\lstset{
   language=JavaScript,
   extendedchars=true,
   basicstyle=\small\ttfamily,
   showstringspaces=false,
   showspaces=false,
   literate=
    {forall}{$\forall$}1
    {->}{$\to$}1
    {=>}{$\implies$}1
    {fun}{$\lambda$}1
}

\newcommand{\N}{\mathbb{N}}
\newcommand{\Z}{\mathbb{Z}}
\newcommand{\Q}{\mathbb{Q}}
\newcommand{\R}{\mathbb{R}}

\usepackage{amsmath}
\usepackage{amsthm}

\usepackage[shortlabels]{enumitem}
\usepackage{mathpartir}
\newcommand{\mif}{\mathsf{if}\ }
\newcommand{\mthen}{\ \mathsf{then}\ }
\newcommand{\melse}{\ \mathsf{else}\ }
\newtheorem{theorem}{Theorem}[section]
\newtheorem{lemma}[theorem]{Lemma}
\theoremstyle{definition}
\newtheorem{definition}{Definition}[section]
\newtheorem{corollary}{Corollary}[theorem]

\newcommand{\nn}{\neg \neg}

\title{Embedding classical logic into constructive logic by double negation translation}
\author{Jules Jacobs}
\begin{document}
\maketitle

\begin{abstract}
  We will embed classical logic into constructive logic via double negation translations. The explanation presented here is intended to be a bit easier to understand and less mysterious than the explanations you'll usually find.\footnote{For instance, on wikipedia (\url{https://en.wikipedia.org/wiki/Double-negation_translation}).}
\end{abstract}

Constructive logic differs from classical logic in that we do not assume $\neg \neg P \implies P$ for all propositions. Proofs that are constructively valid are also classically valid, but not the other way around if the proof uses this principle or an equivalent one like $P \vee \neg P$. Thus, it may seem that the theorems one can prove with classical logic are a superset of those you can prove with constructive logic, but it turns out that from a more refined perspective, the situation is the other way around.

[Describe here that classical proofs of existence give you a method to construct a witness?]

We can translate every classical theorem and proof to some constructive theorem and proof by means of a \emph{double negation translation}. Thus, classical logic may be viewed as the \emph{subset} of constructive logic that lies in the image of this translation. Classical mathematicians restrict themselves to theorems that lie in that image.

Within constructive logic we say that a proposition $P$ is classical if $\neg \neg P \implies P$. The classical propositions form a subset of the constructive ones. Given any $P$, the proposition $\neg \neg P$ is always classical, because $\neg \neg (\neg \neg P) \implies \neg \neg P$, even constructively. The double negation $\neg \neg$ acts as a projection of constructive propositions onto classical propositions.

Using this projection, we can define the classical connectives in terms of the constructive ones:
\begin{align*}
P \wedge^c Q &:= \nn (P \wedge Q) \\
P \vee^c Q &:= \nn (P \vee Q) \\
P \to^c Q & := \nn (P \to Q) \\
\forall^c x, P(x) &:= \nn \forall x, P(x) \\
\exists^c x, P(x) &:= \nn \exists x, P(x) \\
True^c &:= \nn True \\
False^c &:= \nn False \\
\neg^c P & := \nn (\neg P)
\end{align*}
These connectives enjoy all the rules of classical logic, such as $\neg^c (P \wedge^c Q) \iff P \vee^c Q$. Thus, a classical proof of a theorem $T$ can be converted into a constructive proof of theorem $T'$ where $T'$ uses the classical connectives.

Now we can understand the point of view that classical logic is a subset of constructive logic. The classical mathematician can never prove $\exists n\in\N, P(n)$; they only ever prove $\exists^c n\in\N, P(n)$, which is a weaker statement. The former means "I can give you a concrete number $n$ and a proof that $P(n)$ holds", whereas the latter means "If \textbf{you} give me a concrete number $n$ and a proof of $\neg P(n)$, then I can derive a contradiction".

You may think that the difference between those two is not very large. After all, we could simply search for an $n$ satisfying $P(n)$, and this search is guaranteed to terminate if we know $\exists^c n\in\N, P(n)$. This glosses over the fact that to even do the search, we must be able to decide whether $P(n)$ or $\neg P(n)$ for each $n$.

The situation gets worse if the set we are quantifying over is $\R$ instead of $\N$. The constructivist promises to give you a real number, that is, a concrete description of a Cauchy sequence $r_i \in \Q$ with which you can compute arbitrarily accurate rational approximations, and a proof that $P(r)$ holds. The classical mathematician only promises that they can derive a contradiction if \textbf{you} give them such a Cauchy sequence, plus a proof of $\neg P(r)$. Searching over all real numbers for one that works is hopeless, even if we somehow had a magic machine that tells us whether $P(r)$ or $\neg P(r)$ for each particular one.

The definitions of the classical connectives can be simplified. For instance, $\nn True \iff True$, so we might as well define $True^c := True$. The same holds for $False^c$ and $\neg^c$. Another simplication can be had if we assume that \emph{classical connectives are only used on classical propositions}. In that case we can define $P \wedge^c Q := P \wedge Q$, because for classical $P$ and $Q$, we have $\nn (P \wedge Q) \iff P \wedge Q$. The same is true for $\forall^c x, P(x) := \forall x, P(x)$ and $P \to^c Q := P \to Q$.

The same is \emph{not} true for disjunctions and existentials. We do need to retain the $\nn$ around those, even if the propositions involved are classical. This simplifies the translation to:
\begin{align*}
  P \wedge^c Q &:= P \wedge Q \\
  P \vee^c Q &:= \nn (P \vee Q) \\
  P \to^c Q &:= P \to Q \\
  \forall^c x, P(x) &:= \forall x, P(x) \\
  \exists^c x, P(x) &:= \nn \exists x, P(x) \\
  True^c &:= True \\
  False^c &:= False \\
  \neg^c P &:= \neg P
\end{align*}
This still ensures that a formula involving the classical connectives is a classical proposition, and that all the classical laws hold constructively for these connectives, provided that all the basic propositions are classical. We can always \emph{make} the basic propositions classical by putting a $\nn$ around them. For instance, an equality $x = y$ becomes $\nn (x = y)$.

Since $\nn (P \vee Q) \iff \neg (\neg P \wedge \neg Q)$, we could also have used that in the translation. Various different possibilities of shuffling the $\neg$'s around have different names (Gödel-Gentzen's translation, Komolgorov translation, Kuroda's translation). These translations are all constructively equivalent.

\begin{itemize}
  \item How to use the translation to (1) import classical theorems and proofs (2) prove constructive propositions that involve negation by doing the translation in reverse. Decidable propositions.
  \item Expand on Kuroda's translation.
\end{itemize}

\end{document}
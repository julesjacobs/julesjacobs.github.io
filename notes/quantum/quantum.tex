
\documentclass[a4paper, 11pt]{article}
\usepackage[a4paper,left=2.5cm,right=2.5cm,top=2.5cm,bottom=2.5cm]{geometry}
\usepackage[utf8]{inputenc}
\usepackage[T1]{fontenc}
\usepackage{xspace}
\usepackage[bitstream-charter]{mathdesign}
\usepackage{parskip}
\usepackage{microtype}
\usepackage{textcomp}
\usepackage{mathtools}
\usepackage{hyperref}
\usepackage{braket}
\usepackage{tikz}
\usetikzlibrary{quantikz}

\usepackage[all]{hypcap}
\hypersetup{
    colorlinks=true,
    linkcolor=blue,
    filecolor=magenta,
    urlcolor=blue,
}
\usepackage[nameinlink,noabbrev]{cleveref}
\usepackage{float}
\usepackage{listings}
\usepackage{color}
\usepackage{tcolorbox}
\lstdefinelanguage{JavaScript}{
  keywords={typeof, new, true, false, catch, function, return, null, catch, switch, var, while, do, else, case, break, val, then, Definition, Check, Lemma, Proof, Qed, Inductive, Fixpoint, match, for, class, object, extends, override, def, if},
  keywordstyle=\color{blue},
  comment=[l]{//},
  morecomment=[s]{/*}{*/},
  commentstyle=\color{purple}\ttfamily,
  stringstyle=\color{red}\ttfamily
}

\lstset{
   columns=fullflexible,
   language=JavaScript,
   extendedchars=true,
   basicstyle=\small\ttfamily,
   literate=
    % {epsilon}{$\epsilon$}1
    % {empty}{$\emptyset$}1
    % {forall}{$\forall$}1
    % {exists}{$\exists$}1
    % {<->}{$\iff$}1
    {->}{$\to\ $}1
    % {<-}{$\leftarrow\ $}1
    {=>}{$\implies\ $}1
    % {fun}{$\lambda$}1
    % {and}{$\wedge$}1
    % {or}{$\vee$}1
    {cdot}{$\cdot$ }1
}

\newcommand{\N}{\mathbb{N}}
\newcommand{\Z}{\mathbb{Z}}
\newcommand{\Q}{\mathbb{Q}}
\newcommand{\R}{\mathbb{R}}
\newcommand{\C}{\mathbb{C}}

\usepackage{amsmath}
\usepackage{amsthm}

\usepackage[shortlabels]{enumitem}
\setitemize{noitemsep, topsep=1pt, leftmargin=*}
\setenumerate{noitemsep, topsep=1pt, leftmargin=*}
% \setdescription{noitemsep, topsep=0pt, leftmargin=*}
\setdescription{itemsep=.5pt, topsep=0pt, leftmargin=8pt}
\usepackage{mathpartir}

\newtheorem{theorem}{Theorem}[section]
\newtheorem{lemma}[theorem]{Lemma}
\theoremstyle{definition}
\newtheorem{definition}{Definition}[section]
\newtheorem{corollary}{Corollary}[theorem]
\usepackage{adjustbox}
% \usepackage[notref,notcite]{showkeys}
\usepackage{todonotes}
\usepackage{multicol}

% =========== TIKZ ===========
% \usepackage{graphicx}
% \usepackage{tikz}
% \usetikzlibrary{shapes.geometric, arrows}
% \usetikzlibrary{fit}
% \usetikzlibrary{svg.path}
% % \usetikzlibrary{graphdrawing}
% % \usetikzlibrary{graphdrawing.force}
% % \usetikzlibrary{graphdrawing.layered}
% \usetikzlibrary{decorations}
% \usetikzlibrary{decorations.markings}
% \usetikzlibrary{backgrounds}

\newcommand{\ie}{\textit{i.e.,}\xspace}

\usepackage{lipsum}

\title{A quick introduction to quantum programming}
\author{Jules Jacobs}

\begin{document}
\maketitle

\begin{abstract}
  This note is a quick introduction to quantum programming in the circuit model.
  A quantum computer on $k$ bits gets as input a \emph{quantum circuit description},
  and produces as output a random string of $k$ bits according to a probability distribution determined by the quantum circuit.
  A quantum programming language in this model is a language for creating such quantum circuits.

  This note contains a quick but formal introduction to these concepts.
  After reading it, you will be able to write a computer program that simulates such a quantum computer
  (albeit exponentially more slowly than an actual quantum computer would execute a circuit, which is the point!).
\end{abstract}

\section{Quantum states}

Imagine that we have a box with some physical system inside of it, with a finite set $S$ of possible states.
A probability distribution over $S$ is a vector $\vec{p}$ of probabilities, one probability $p_x \in [0,1]$ for each state $x \in S$, such that $\sum_x p_x = 1$.

A \emph{quantum state} over $S$, on the other hand, is a vector $\vec{\phi}$ of \emph{probability amplitudes}, one complex number $\phi_x \in \C$ for each state $x \in S$.
If we \emph{measure} such a quantum state, we obtain outcome $x \in S$ with probability $p_x = |\phi_x|^2$.
Thus, in order for $\phi$ to be a proper quantum state, we must have $\sum_x |\phi_x|^2 = 1$.

\section{Time evolution in quantum mechanics}

Imagine that the system in the box evolves in time according to some laws of physics.
In quantum mechanics, the state evolution is given by a matrix $U$ that multiplies the state every time step.
If the state is currently $\phi$, then at the next time step the state is $U\phi$.
If there are $n = |S|$ possible states, then $U$ is an $n \times n$ matrix.
Only matrices that preserve the condition that the probabilities sum to $1$ are allowed: if $\sum_x |\phi_x|^2 = 1$ we must have $\sum_x |(U\phi)_x|^2 = 1$.
Such matrices are called \emph{unitary}.

It might be helpful to compare with probabilistic evolution of the state as in a Markov chain.
In that case we model the state with a probability vector $\vec{p}$ and we multiply this vector with a matrix $M$ at each time step.
If the state is currently $p$, then at the next time step the state is $Mp$.
Matrices that preserve the condition that all probabilities are non-negative and that their sum remains $1$ are called \emph{stochastic matrices}.
The entry $M_{x,y}$ of the matrix is the probability that the system will step to state $y$, if the state is currently $x$.
In the quantum case, the entry $U_{x,y}$ of the unitary matrix, is the \emph{probability amplitude} of next state being $y$, if the state is currently $x$.

\section{What a quantum computer is}

A quantum computer with state set $S$ is a device where we can \emph{input} such a matrix $U$ and an initial state $x \in S$.
It will then do one step of time evolution using $U$, and it will \emph{measure} the new state and tell us which outcome $y \in S$ it got.
Thus, a quantum computer is a kind of universal quantum mechanics simulator:
\begin{enumerate}
  \item We \emph{input} the initial state $x \in S$ and matrix $U$
  \item The quantum computer \emph{outputs} answer $y \in S$ with probability $|U_{x,y}|^2$
\end{enumerate}

We will refine this description in the next section.

\section{Quantum circuits}

In physics, the state set $S$ is often infinite, and sometimes even uncountably infinite (e.g. the position of a particle),
but in quantum programming the set $S = \{0,1\}^k$ is taken to be strings of $k$ bits, so that $|S| = 2^k$.
Still, this means that $U$ is a $2^k$-by-$2^k$ matrix. One might wonder how we even input the $U$ to the quantum computer, if it contains an exponential amount of data.

The answer is that we can't quite input \emph{any} matrix $U$; it must be encoded as a \emph{quantum circuit}.
A quantum circuit is a sequence of unitary operations we do on the state of $n$ bits, where each operation operates on some small subset of the bits and leaves the rest of the bits alone.

Often, a small set of primitive operations is used, such as the \emph{Hadamard gate} and the \emph{CCNOT gate}.
The Hadamard gate operates on one bit, and the CCNOT gate operates on three bits.

In order to describe what they do, we introduce a bit of notation for \emph{definite states}.
We use the notation $\phi = \ket{01001}$ for the definite state $\phi$ where $\phi_{01001} = 1$ and $\phi_x = 0$ otherwise, \ie the state that puts all probability amplitude on $01001$.
Because gates $A$ are linear operators, it is sufficient to define their behavior on definite states $\ket{x}$ for $x \in \{0,1\}^k$,
since for a superposition $\phi = \sum_x \phi_x \ket{x}$ we have $A\phi = \sum_x \phi_x A \ket{x}$.

\subsection{The Hadamard gate}

\newcommand{\Ha}{H}
\newcommand{\CCNOT}{\mathsf{CCNOT}}

The Hadamard gate $\Ha$ operates on one bit, and is defined as:
\begin{align*}
  \Ha\ket{0} = \frac{1}{\sqrt{2}}(\ket{0} + \ket{1}) \\
  \Ha\ket{1} = \frac{1}{\sqrt{2}}(\ket{0} - \ket{1})
\end{align*}
Equivalently, we can define it using matrix notation, as
\begin{align*}
  \Ha = \frac{1}{\sqrt{2}}\begin{pmatrix}
    1 & 1 \\
    1 & -1
  \end{pmatrix}
\end{align*}

If we have $n$ bits in the state, then we have Hadamard gates $\Ha_1, \Ha_2, \cdots, \Ha_n$, each operating on a different bit.
This is what $\Ha_1$ does:
\begin{align*}
  \Ha_1 \ket{0 b_1 b_2 \cdots b_n} = \frac{1}{\sqrt{2}}(\ket{0 b_1 b_2 \cdots b_n} + \ket{1 b_1 b_2 \cdots b_n}) \\
  \Ha_1 \ket{1 b_1 b_2 \cdots b_n} = \frac{1}{\sqrt{2}}(\ket{0 b_1 b_2 \cdots b_n} - \ket{1 b_1 b_2 \cdots b_n})
\end{align*}
Try writing down $H_1$ as a $2^n$-by-$2^n$ matrix, and you'll see why this notation is useful.

\subsection{Classical gates}

Given a function $f : \{0,1\}^k \to \{0,1\}^k$ on bit strings of length $k$, we define a classical gate $C_f$ given by:
\begin{align*}
  C_f \ket{x} = \ket{f(x)}
\end{align*}
That is, given a definite state $\ket{x}$, it outputs another definite state $\ket{f(x)}$.
The function $f$ must be bijective in order for this to be a unitary operator.

A common trick is to start with a boolean function $g : \{0,1\}^{k-1} \to \{0,1\}$ and define
\begin{align*}
  f(b_1,\cdots,b_{k-1},b_k) = (b_1,\cdots,b_{k-1}, b_k \oplus g(b_1,\cdots,b_{k-1}))
\end{align*}
This function $f$ leaves the first $k-1$ bits alone, and XORs the last bit with $g(b_1,\cdots,b_{k-1})$.
You may verify that $f$ is always bijective.

The CCNOT gate is an example of this: we take $g(b_1,b_2)$ to be the local AND of $b_1$ and $b_2$.
This gives us a gate that operates on three bits, which leaves the first two bits alone and flips the third bit if the first two bits are both $1$.
The CCNOT gate is a universal classical gate: any bijection $f$ can be built out of CCNOT gates (analogous to how any boolean function can be built out of NAND gates).
As with Hadamard gates, we can apply a CCNOT gate to any three bits in the state, and thus we have $\CCNOT_{ijk}$ gates,
which applies the gate to bits $i,j,k$. This is what $\CCNOT_{123}$ does:
\begin{align*}
  \CCNOT_{123} \ket{b_1 b_2 b_3 b_4 \cdots b_k} = \ket{b_1 b_2 (b_3 \oplus (b_1 \wedge b_2)) b_4 \cdots b_k}
\end{align*}
That is, it flips the third bit if the first two bits are $1$, and leaves the other bits alone.

\subsection{Composition of gates}

We input the matrix $U$ into the quantum computer as a sequence of operations, e.g.
\begin{align*}
  U = H_2 \cdot H_3 \cdot \CCNOT_{234} \cdot H_2 \cdot H_4 \cdot \CCNOT_{123} \cdot H_1 \cdot H_2
\end{align*}
We can graphically represent this circuit as follows:
\\
\begin{center}
\begin{quantikz}
  & \gate{H} & \gate[wires=3][2cm]{CCNOT} & \qw & \qw & \qw & \qw \\
  & \gate{H} & & \gate{H} & \gate[wires=3][2cm]{CCNOT} & \gate{H} & \qw \\
  & \qw & \qw & \qw & & \gate{H} & \qw \\
  & \qw & \qw & \gate{H} & & \qw & \qw \\
\end{quantikz}
\end{center}

The Hadamard gate and CCNOT gate together are universal, in the sense that any quantum computation can be done using only these two gates.\footnote{This more or less means that any unitary matrix can be approximated as product of Hadamard gates and CCNOT gates.
Since these gates all have real valued matrices, this is not exactly true, see \cite{aharonovSimpleProofThat2003} for the technical sense in which these two gates are universal.}

\section{The Deutsch–Jozsa algorithm}

TODO

\section{A quantum circuit simulator}

TODO

\section{Quantum programming languages}

A quantum programming language is a high level language for assembling quantum circuits.
These languages often include features that allow you to take classical computation and turn them into a quantum circuit (e.g. using a large number of CNOT gates),
and some languages go far beyond this in terms of feature set.
See \url{https://en.wikipedia.org/wiki/Quantum_programming} for a list of quantum programming languages.

\bibliographystyle{alphaurl}
\bibliography{quantum}


\end{document}

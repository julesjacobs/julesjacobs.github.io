\documentclass[a4paper, 11pt]{scrartcl}
\usepackage[utf8]{inputenc}
\usepackage[scaled=0.88]{beraserif}
\usepackage[scaled=0.85]{berasans}
\usepackage[scaled=0.84]{beramono}
\usepackage[dottedtoc]{classicthesis}
% \clearscrheadfoot
% \ohead[]{\headmark}
% \cfoot[\pagemark]{\pagemark}
\usepackage[a4paper,left=2.5cm,right=2.5cm,top=2.5cm,bottom=2.5cm]{geometry}
\usepackage[T1]{fontenc}
\usepackage{mathpazo}
\linespread{1.05}
\usepackage[T1,small,euler-digits]{eulervm}
\setkomafont{disposition}{}
\setkomafont{section}{}
\newcommand{\sectioncolor}{black}
\newcommand{\sectionnumbercolor}{Maroon!60}
\titleformat{\section}
            {\color{\sectioncolor}\large\usekomafont{disposition}\usekomafont{section}}
            {\color{\sectionnumbercolor}\llap{\textsc{\MakeTextLowercase{\thesection}}\hspace{0.7em}}}
            {0pt}
            {\usekomafont{disposition}\usekomafont{section}\spacedlowsmallcaps}
            % [{\color{Maroon}\titlerule}]

\titleformat{\subsection}
            {\color{\sectioncolor}\large\usekomafont{disposition}\usekomafont{section}}
            {\color{\sectionnumbercolor}\llap{\textsc{\MakeTextLowercase{\thesubsection}}\hspace{0.7em}}}
            {0pt}
            {\usekomafont{disposition}\usekomafont{section}\spacedlowsmallcaps}

\pagestyle{plain}

% \usepackage[T1]{fontenc}
% \usepackage[bitstream-charter]{mathdesign}

% \RequirePackage[T1]{fontenc}
% \RequirePackage[tt=false, type1=true]{libertine}
% \RequirePackage[varqu]{zi4}
% \RequirePackage[libertine]{newtxmath}

\usepackage{xspace}
\usepackage{parskip}
\usepackage{microtype}
\usepackage{textcomp}
\usepackage{mathtools}
\usepackage{hyperref}
\usepackage[all]{hypcap}
\hypersetup{
    % colorlinks=true,
    % linkcolor=blue,
    % filecolor=magenta,
    % urlcolor=blue,
    linktoc=all
}
\usepackage[nameinlink,noabbrev]{cleveref}
\usepackage{float}
\usepackage{listings}
\usepackage{color}
\usepackage[dvipsnames]{xcolor}
\usepackage{tcolorbox}

\lstdefinelanguage{Python}{
  keywords={from, import, def, for, in, if, return, lambda},
  comment=[l]{\#},
  morestring=[b]',
  morestring=[b]",
}

\lstset{
   columns=fullflexible,
   language=Python,
  %  keywordstyle=\color{RoyalBlue!30!black},
  %  keywordstyle=\color{Maroon},
   keywordstyle=\bfseries\color{Maroon!65},
   commentstyle=\color{Green!80!black}\ttfamily,
   identifierstyle=\color{RoyalBlue!70!black},
   stringstyle=\color{red!50!black}\ttfamily,
   showstringspaces=false,
   extendedchars=true,
   basicstyle=\small\ttfamily,
   % breaklines=true,
}


\newcommand{\N}{\mathbb{N}}
\newcommand{\Z}{\mathbb{Z}}
\newcommand{\Q}{\mathbb{Q}}
\newcommand{\R}{\mathbb{R}}
\newcommand{\C}{\mathbb{C}}

\usepackage{amsmath}
\let\openbox\undefined
\usepackage{amsthm}

\usepackage[shortlabels]{enumitem}
\setitemize{noitemsep, topsep=1pt, leftmargin=*}
\setenumerate{noitemsep, topsep=1pt, leftmargin=*}
% \setdescription{noitemsep, topsep=0pt, leftmargin=*}
\setdescription{itemsep=.5pt, topsep=0pt, leftmargin=8pt}
\usepackage{mathpartir}

\newtheorem{theorem}{Theorem}[section]
\newtheorem{lemma}[theorem]{Lemma}
\theoremstyle{definition}
\newtheorem{definition}{Definition}[section]
\newtheorem{corollary}{Corollary}[theorem]
\usepackage{adjustbox}
% \usepackage[notref,notcite]{showkeys}
\usepackage{todonotes}
\usepackage{multicol}

% =========== TIKZ ===========
% \usepackage{graphicx}
% \usepackage{tikz}
% \usetikzlibrary{shapes.geometric, arrows}
% \usetikzlibrary{fit}
% \usetikzlibrary{svg.path}
% % \usetikzlibrary{graphdrawing}
% % \usetikzlibrary{graphdrawing.force}
% % \usetikzlibrary{graphdrawing.layered}
% \usetikzlibrary{decorations}
% \usetikzlibrary{decorations.markings}
% \usetikzlibrary{backgrounds}
\newcommand\blfootnote[1]{%
  \begingroup
  \renewcommand\thefootnote{}\footnote{#1}%
  \addtocounter{footnote}{-1}%
  \endgroup
}
\newcommand{\ie}{\textit{i.e.,}\xspace}

\makeatletter
\renewcommand\maketitle
  {
  \begin{center}
   {\color{black}\large\spacedallcaps{\@title}}\\\bigskip%
   \large{\@author}\\\medskip%
   \normalsize{\@date}\bigskip%
  \end{center}%
  }

\newcommand{\tac}[1]{\lstinline[mathescape]~#1~}
\newcommand{\ciff}{\ \leftrightarrow\ }
\newcommand{\hyp}{\tac{H}}
\newcommand{\hypB}{\tac{G}}
\newcommand{\var}{\tac{x}}
\newcommand{\varB}{\tac{y}}

\title{Coq cheatsheet}
\author{Jules Jacobs}

\begin{document}
\maketitle

% \tableofcontents

\section{Introduction}

This is Coq code that proves the strong induction principle for natural numbers:

\lstinputlisting[language=Coq]{lemma.v}

Coq proofs manipulate the \emph{proof state} by executing a sequence of \emph{tactics} such as \tac{intros}, \tac{eapply}, \tac{induction}.
Coq calculates the proof state for you after executing each tactic.
Here's what Coq displays after executing the second \lstinline|intros m Hm.|:

\begin{minipage}{\textwidth}
\lstinputlisting[language=Coq]{proofstate.txt}
\end{minipage}

The proof state consists of a list of variables and hypotheses above the line, and a goal below the line.
Executing a tactic may result in zero or more subgoals. A subgoal is solved if we succesfully apply a tactic that creates no new subgoals (such as the \tac{lia} tactic, which solves simple numeric goals).
Some tactics create multiple subgoals, such as the \tac{induction} tactic: it creates one subgoal for the base case of the induction, and one subgoal for the inductive case.
% \footnote{Coq allows us to do induction not only on natural numbers, but also on other data types. Induction on other data types may create any number of subgoals, one for each constructor of the data type.}
If a tactic creates multiple subgoals, we solve them using a bulleted list of tactic scripts, or using brackets:

\vspace{-0.5cm}
\begin{minipage}[t]{0.3\textwidth}
\begin{lstlisting}
  (* Simple bullets *)
  tac1.
  + tac2.
  + tac3.
  + tac4.
  + tac5.
  + tac6.
\end{lstlisting}
\end{minipage}
\begin{minipage}[t]{0.3\textwidth}
\begin{lstlisting}
  (* Nested bullets *)
  tac1.
  + tac2.
     * tac3
     * tac4.
  + tac5.
    ++ tac6.
\end{lstlisting}
\end{minipage}
\begin{minipage}[t]{0.3\textwidth}
\begin{lstlisting}
  (* Brackets *)
  tac1.
  { tac2. }
  { tac3. }
  tac4.
  { tac5. }
  tac6.
\end{lstlisting}
\end{minipage}

We usually use bullets if the subgoals are on equal footing, and we use brackets for simple side-conditions.
We do not have to enclose the last subgoal in brackets, thus preventing deep nesting.

\section{Logical reasoning}

% We divide the logical reasoning tactics into those that modify the goal and those that modify a hypothesis.

\subsection{Tactics that modify the goal}

\begin{tabular}{c l}
  \textbf{Goal} & \textbf{Tactic} \\ \midrule
  $P \to Q$ & \tac{intros H} \\
  $\neg P$ & \tac{intros H} \quad (Coq defines $\neg P$ as $P \to False$) \\
  $\forall x, P(x)$ & \tac{intros x} \\
  $\exists x, P(x)$ & \tac{exists x}, \tac{eexists} \\
  $P \land Q$ & \tac{split} \quad (also works for $P \leftrightarrow Q$, which is defined as $(P \to Q) \land (Q \to P)$)\\
  $P \lor Q$ & \tac{left}, \tac{right} \\
  $Q$ & \tac{apply H}, \tac{eapply H} (where \tac{H} $: (...) \to Q$ is a lemma with conclusion $Q$) \\
  $False$ & \tac{apply H}, \tac{eapply H} (where \tac{H} $: (...) \to \neg P$ is a lemma with conclusion $\neg P$) \\
  Any goal & \tac{exfalso} \quad (turns any goal into $False$) \\
  Skip goal & \tac{admit} \quad (skips goal so that you can work on other subgoals)\\ \midrule
\end{tabular}

When using \tac{apply H} with a lemma \hyp\ $: P_1 \to P_2 \to Q$, Coq will create subgoals for each assumption $P_1$ and $P_2$.
If the lemma has no assumptions, then then \tac{apply H} finishes the goal.

When using \tac{apply H} with a quantified lemma \hyp\ $: \forall x, (...)$, Coq will try to automatically find the right $x$ for you.
The \tac{apply} tactic will fail if Coq cannot determine $x$.
You can then explicitly choose an instantiation $x = 4$ using \tac{apply (H 4)}.
You can also use \tac{eapply H} to use an E-var $?x$, which means that the instanation will be determined later.
If there are multiple $\forall$-quantifiers you can do \tac{eapply (H _ _ 4 _)}, to let Coq determine the ones where you put $\_$.

Similarly, \tac{eexists} will instantiate an existential quantifier with an E-var.
If your goal is $\exists n, P\ n$ and you have \hyp\ $: P\ 3$, then you can type \tac{eexists. apply H.} This automatically determines $n=3$.
% Coq uses \emph{unification} to determine the instantiation.
% The unification algorithm can also reason modulo certain equations.
% This is necessarily a heuristic method, because the problem is undecidable.
% When you run into issues, try giving the instantiation explicitly.

\subsection{Tactics that modify a hypothesis}
\vspace{-0.2cm}

\begin{tabular}{l l}
  \textbf{Hypothesis} & \textbf{Tactic} \\ \midrule
  \hyp\ $: False$ & \tac{destruct H} \\
  \hyp\ $: P \land Q$ & \tac{destruct H as [H1 H2]} \\
  \hyp\ $: P \lor Q$ & \tac{destruct H as [H1|H2]} \\
  \hyp\ $: \exists x, P(x)$ & \tac{destruct H as [x H]} \\
  \hyp\ $: \forall x, P(x)$ & \tac{specialize (H y)}\\
  \hyp\ $: P \to Q$ & \tac{specialize (H G)} \quad (where \hypB\ $: P$ is a lemma or hypothesis) \\
  \hyp\ $: P$ & \tac{apply G in H}, \tac{eapply G in H} \quad (where \hypB\ $: P \to (...)$ is a lemma or hypothesis) \\
  \hyp\ $: P$, \var\ $: A$ & \tac{clear H}, \tac{clear x} \quad (remove hypothesis \hyp\ or variable \var) \\ \midrule
\end{tabular}

\subsection{Forward reasoning}
\vspace{-0.2cm}

\begin{tabular}{l l}
  \textbf{Tactic} & \textbf{Meaning} \\ \midrule
  \tac{assert P as H} & Create new hypothesis \hyp\ $: P$ after proving subgoal $P$ \\
  \tac{assert P as H by tac} & Create new hypothesis \hyp\ $: P$ after proving subgoal $P$ using \tac{tac} \\
  \tac{assert (G := H)} & Duplicate hypothesis \\
  \tac{cut P} & Split goal $Q$ into two subgoals $P \to Q$ and $P$\\ \midrule
  % \tac{wlog} & ... \\
\end{tabular}

Brackets are useful with the assert tactic:
\tac{assert P as H. \{ ... proof of P ... \}}

\newpage
\section{Equality, rewriting, and computation rules}

\begin{tabular}{l l}
  \textbf{Tactic} & \textbf{Meaning} \\ \midrule
  \tac{reflexivity} & Solve goal of the form \tac{x = x} or \tac{P $\ciff$ P} \\
  \tac{symmetry} & Turn goal \tac{x = y} into \tac{y = x} (or \tac{P $\ciff$ Q}) \\
  \tac{symmetry in H} & Turn hypothesis \tac{H : x = y} into \tac{H : y = x} (or \tac{P $\ciff$ Q})\\ \midrule
  \tac{unfold f} & Replace constant \tac{f} with its definition (only in the goal) \\
  \tac{unfold f in H} & Replace constant \tac{f} with its definition (in hypothesis \tac{H}) \\
  \tac{unfold f in *} & Replace constant \tac{f} with its definition (everywhere) \\ \midrule
  \tac{simpl} & Rewrite with computation rules (in the goal) \\
  \tac{simpl in H} & Rewrite with computation rules (in hypothesis \tac{H}) \\
  \tac{simpl in *} & Rewrite with computation rules (everywhere) \\ \midrule
  \tac{rewrite H.} & Rewrite \tac{H : x = y} or \tac{H : P $\ciff$ Q} (in the goal). \\
  \tac{rewrite H in G.} & Rewrite \tac{H} (in hypothesis \tac{G}). \\
  \tac{rewrite H in *.} & Rewrite \tac{H} (everywhere). \\ \midrule
  \tac{rewrite <-H.} & Rewrite \tac{H : x = y} backwards. \\
  \tac{rewrite H,G.} & Rewrite using \tac{H} and then \tac{G}. \\
  \tac{rewrite !H.} & Repeatedly rewrite using \tac{H}. \\
  \tac{rewrite ?H.} & Try rewriting using \tac{H}. \\ \midrule
  \tac{subst} & Substitute away all equations \tac{H : x = A} with a variable on one side. \\
  \tac{injection H as H} & Use injectivity of \tac{C} to turn \tac{H : C x = C y} into \tac{H : x = y}. \\
  \tac{discriminate H} & Solve goal with inconsistent assumption \tac{H : C x = D y}. \\
  \tac{simplify\_eq} & Automated tactic that does \tac{subst}, \tac{injection}, and \tac{discriminate} automatically. \\ \midrule
\end{tabular}

Rewriting also works with quantified equalities.
If you have $H : \forall n m, n + m = m + n$ then you can do \tac{rewrite H}.
Coq will instantiate $n$ and $m$ based on what it finds in the goal.
You can specify a particular instantiation $n=3, m=5$ using \tac{rewrite (H 3 5).}

The \tac{simplify\_eq} tactic is from \href{https://plv.mpi-sws.org/coqdoc/stdpp/stdpp.tactics.html}{stdpp}.
Although it is not a built-in tactic, I mention it because it is incredibly useful.

\newpage
\section{Inductive types and relations}

\subsection{Inductive types $Foo$}

\begin{tabular}{c l}
  \textbf{Term} & \textbf{Tactic} \\ \midrule
  $x : Foo$ & \tac{destruct x as [a b|c d e|f]} \\
  $x : Foo$ & \tac{destruct x as [a b|c d e|f] eqn:E} \quad (adds equation \tac{E : x = (...)} to context)\\
  $x : Foo$ & \tac{induction x as [a b IH|c d e IH1 IH2|f IH]} \\ \midrule
\end{tabular}

\subsection{Inductive relations $Foo\ x\ y$}

\begin{tabular}{c l}
  \textbf{Goal/Hypothesis} & \textbf{Tactic} \\ \midrule
  $Foo\ x\ y$ & \tac{constructor}, \tac{econstructor} \quad (tries applying all constructors of $Foo$) \\
  $H : Foo\ x\ y$ & \tac{inversion H} \quad (use when $x$,$y$ are fixed terms) \\
  $H : Foo\ x\ y$ & \tac{induction H} \quad (use when $x$,$y$ are variables)\\  \midrule
\end{tabular}

It is often useful to define the tactic \tac{Ltac inv H := inversion_clear H; subst.} and use this instead of \tac{inversion}.

\subsection{Getting the right induction hypothesis}
The \tac{revert} tactic is useful to obtain the correct induction hypothesis:

\begin{tabular}{c l}
\textbf{Hypothesis} & \textbf{Tactic} \\ \midrule
$H : P$ & \tac{revert H} \quad (opposite of \tac{intros H}: turn goal $Q$ into $P \to Q$) \\
$x : A$ & \tac{revert x} \quad (opposite of \tac{intros x}: turn goal $Q$ into $\forall x, Q$) \\ \midrule
\end{tabular}

A common pattern is \tac{revert x. induction n; intros x; simpl.}
A good rule of thumb is that you should create a separate lemma for each inductive argument, so that \tac{induction} is only ever used at the start of a lemma (possibly preceded by some \tac{revert}).

\section{Proof search with \texttt{eauto}}

The \tac{eauto} tactic tries to solve goals using \tac{eapply}, \tac{reflexivity}, \tac{eexists}, \tac{split}, \tac{left}, \tac{right}.
You can specify the search depth using \tac{eauto n} (the default is $n=5$).

You can give \tac{eauto} additional lemmas to use with \tac{eauto using lemma1,lemma2}.
You can also use \tac{eauto using foo} where \tac{foo} is an inductive type. This will use all the constructors of \tac{foo} as lemmas.

\newpage
\section{Intro patterns}

The \tac{destruct x as pat} and \tac{intros pat} tactics can unpack multiple levels at once using nested \emph{intro patterns}:
if the goal is $(P \land \exists x : option\ A, Q_1 \lor Q_2) \to (...)$ then \tac{intros [H [[x|] [G|G]]]}
splits the conjunction, unpacks the existential, case analyzes the $x : option\ A$, and case analyzes the disjunction (creating 4 subgoals).
The intros tactic can also be chained to introduce multiple hypotheses: \tac{intros x y.} $\equiv$ \tac{intros x. intros y.}

\begin{tabular}{c l}
  \textbf{Data} & \textbf{Pattern} \\ \midrule
  $\exists x, P$ & \tac{[x H]} \\
  $P \land Q$ & \tac{[H1 H2]} \\
  $P \lor Q$ & \tac{[H1|H2]} \\
  $False$ & \tac{[]} \\ \midrule
  $A * B$ & \tac{[x y]} \\
  $A + B$ & \tac{[x|y]} \\
  $option\ A$ & \tac{[x|]} \\
  $bool$ & \tac{[|]} \\
  $nat$ & \tac{[|n]} \\
  $list\ A$ & \tac{[x xs|]} \\
  Inductive type & \tac{[a b|c d e|f]} \\
  Inductive type & \tac{[]}\quad (unpack with names chosen by Coq) \\ \midrule
  $x = y$ & \tac{->} \quad (substitute the equality $x \mapsto y$)\\
  $x = y$ & \tac{<-} \quad (substitute the equality $y \mapsto x$)\\ \midrule
  Any & \tac{?}\quad (introduce variable/hypothesis with name chosen by Coq) \\ \midrule
\end{tabular}

Furthermore, \tac{(x \& y \& z \& ...)} is equivalent to \tac{[x [y [z ...]]]}.

Because $\exists x, P$, $P \land Q$, $P \lor Q$, $False$ are \emph{defined} as inductive types, their intro patterns are special cases of the intro pattern for inductive types, and you can also use the \tac{[]} intro pattern for them.

Intro patterns can be used with the \tac{assert P as pat} tactic,
e.g. \tac{assert (A = B) as ->} or \tac{assert (exists x, P) as [x H].}
You can also use them with the \tac{apply H in G as pat} tactic.

\section{Composing tactics}

\begin{tabular}{l l}
  \textbf{Tactic} & \textbf{Meaning} \\ \midrule
  \tac{tac1; tac2} & Do \tac{tac2} on all subgoals created by \tac{tac1}. \\
  \tac{tac1; [tac2|..]} & Do \tac{tac2} only on the first subgoal. \\
  \tac{tac1; [..|tac2]} & Do \tac{tac2} only on the last subgoal. \\
  \tac{tac1; [tac2|..|tac3|tac4]} & Do tactics on corresponding subgoals. \\
  \tac{tac1; [tac2|tac3..|tac4]} & Do tactics on corresponding subgoals. \\
  \tac{tac1 || tac2} & Try \tac{tac1} and if it fails do \tac{tac2}. \\
  \tac{try tac1} & Try \tac{tac1}, and do nothing if it fails. \\
  \tac{repeat tac1} & Repeatedly do \tac{tac1} until it fails. \\
  \tac{progress tac1} & Do \tac{tac1} and fail if it does nothing. \\ \midrule
\end{tabular}

In the examples above, the two dots are part of the Coq syntax.

\section{Searching for lemmas and definitions}

\begin{tabular}{l l}
  \textbf{Command} & \textbf{Meaning} \\ \midrule
  \tac{Search nat.} & Prints all lemmas and definitions about \tac{nat}. \\
  \tac{Search (0 + _ = _).} & Prints all lemmas containing the pattern \tac{0 + _ = _}. \\
  \tac{Search (_ + _ = _) 0.} & Prints all lemmas containing \tac{_ + _ = _} and \tac{0}. \\
  \tac{Search (list _ -> list _).} & Prints all definitions and lemmas containing the pattern. \\
  \tac{Search Nat.add Nat.mul.} & Prints all lemmas relating addition and multiplication. \\
  \tac{Search "rev".} & Prints all definitions and lemmas containing substring "rev". \\
  \tac{Search "+"$\ $"*"$\ $"=".} & Prints all definitions and lemmas containing both\ \ \tac{+, *, =}. \\ \midrule
  \tac{Check (1+1).} & Prints the type of \tac{1+1} \\
  \tac{Compute (1+1).} & Prints the normal form of \tac{1+1}. \\
  \tac{Print Nat.add.} & Prints the definition of \tac{Nat.add} \\
  \tac{About Nat.add.} & Prints information about \tac{Nat.add}. \\
  \tac{Locate "+".} & Prints information about notation "+". \\ \midrule
\end{tabular}

\section{Examples of custom tactics}

\lstinputlisting[language=Coq]{tactics.v}
\url{http://julesjacobs.com/notes/coq-cheatsheet/tactics.v}

% \section{Todo}

% \begin{itemize}
%   \item Installing Coq
%   \item Compilation and multiple files
%   \item Common error messages
%   \item Definition, Fixpoint, Inductive
%   \item Records
%   \item Implicit arguments
%   \item Hint databases
%   \item match\_goal
%   \item Type classes
%   \item setoid\_rewrite
%   \item CoInductive, cofix (and fix)
%   \item Mutually inductive lemmas
%   \item ssreflect
%   \item stdpp
%   \item Modules
%   \item Unicode
% \end{itemize}

% \texttt{julesjacobs@gmail.com}

% \section{Advanced}

% \subsection{Hint databases}

% TODO

% \subsection{Type classes}

% TODO

% \subsection{Match goal}

% TODO

% \subsection{Useful higher level tactics}

% From Coq itself:

% \begin{itemize}
%   \item \tac{lia}
%   \item \tac{setoid\_rewrite}
%   \item \tac{symmetry}, \tac{transitivity}, \tac{etransitivity}
%   \item \tac{congruence}
%   \item \tac{f\_equal}, \tac{f\_equiv}
% \end{itemize}

% From stdpp:

% \begin{itemize}
%   \item \tac{done}, \tac{by}
%   \item \tac{simplify\_eq}
%   \item \tac{naive\_solver}
%   \item \tac{set\_solver}
%   \item \tac{case\_decide}, \tac{case\_match}
%   \item For more, see \url{https://plv.mpi-sws.org/coqdoc/stdpp/stdpp.tactics.html}
% \end{itemize}

% Another useful aspect of stdpp it that it defines hint databases for several tactics, so that you can use them with \tac{eauto}, e.g. \tac{eauto with lia}.

\end{document}
\documentclass[a4paper, 11pt]{article}
\usepackage[a4paper,left=2.3cm,right=2.3cm,top=2.5cm,bottom=2.5cm]{geometry}
\usepackage[utf8]{inputenc}
\usepackage[T1]{fontenc}
\usepackage[bitstream-charter]{mathdesign}
\usepackage{microtype}
\usepackage{textcomp}

\usepackage{mathtools}
\newcommand\SetSymbol[1][]{\nonscript\:#1\vert\allowbreak\nonscript\:\mathopen{}}
\providecommand\given{} % to make it exist
\DeclarePairedDelimiterX\Set[1]\{\}{\renewcommand\given{\SetSymbol[\delimsize]}#1}

\usepackage{hyperref}
\hypersetup{
    colorlinks=true,
    linkcolor=blue,
    filecolor=magenta,
    urlcolor=blue,
}
\usepackage{listings}
\usepackage{color}
\definecolor{lightgray}{rgb}{.9,.9,.9}
\definecolor{darkgray}{rgb}{.4,.4,.4}
\definecolor{purple}{rgb}{0.65, 0.12, 0.82}

\lstdefinelanguage{JavaScript}{
  keywords={typeof, new, true, false, catch, function, return, null, catch, switch, var, in, while, do, else, case, break, val, then, Inductive, Fixpoint, match, end},
  keywordstyle=\color{blue},
  ndkeywords={class, export, boolean, throw, implements, import, this},
  ndkeywordstyle=\color{darkgray}\bfseries,
  identifierstyle=\color{black},
  sensitive=false,
  comment=[l]{//},
  morecomment=[s]{/*}{*/},
  commentstyle=\color{purple}\ttfamily,
  stringstyle=\color{red}\ttfamily
}

\lstset{
   language=JavaScript,
   extendedchars=true,
   basicstyle=\small\ttfamily,
   showstringspaces=false,
   showspaces=false,
   literate=
    {forall}{$\forall$}1
    {->}{$\to$}1
    {=>}{$\implies$}1
    {fun}{$\lambda$}1
}

\newcommand{\N}{\mathbb{N}}
\newcommand{\Z}{\mathbb{Z}}
\newcommand{\Q}{\mathbb{Q}}
\newcommand{\R}{\mathbb{R}}

\usepackage{amsmath}
\usepackage{amsthm}

\usepackage[shortlabels]{enumitem}
\usepackage{mathpartir}
\newcommand{\mif}{\mathsf{if}\ }
\newcommand{\mthen}{\ \mathsf{then}\ }
\newcommand{\melse}{\ \mathsf{else}\ }
\newtheorem{theorem}{Theorem}[section]
\newtheorem{lemma}[theorem]{Lemma}
\theoremstyle{definition}
\newtheorem{definition}{Definition}[section]
\newtheorem{corollary}{Corollary}[theorem]


\title{Determinant formulas for symmetric polynomials of eigenvalues}
\author{Jules Jacobs}
\begin{document}
\maketitle

\begin{abstract}
 Abstract
\end{abstract}

\section{Introduction}

\begin{theorem}
  Let $\mathbb{K}$ be a finite index set, and let $A^{(k)}$ be $n\times n$ matrices for $k \in \mathbb{K}$.
  Then the quantity
  \begin{align}
    \sum_{K\in \mathbb{K}^{n\times n}} p_K \det_{ij}(A^{(K_{ij})}_{ij})
  \end{align}
  is independent of the basis of the $A^{(k)}$ if $p_K$ is symmetric (i.e. $p_K = p_{K'}$ if $K'$ is the same as $K$ up to a row and column permutation).
\end{theorem}

\begin{theorem}
  Let $\mathbb{K}$ be a finite index set, and let $A^{(k)}$ be commuting $n\times n$ matrices with eigenvalues $a^{(k)}_i$ for $k \in \mathbb{K}$,
  then
  \begin{align}
    \sum_{K\in \mathbb{K}^{n\times n}} p_K \det_{ij}\left(A^{(K_{ij})}_{ij}\right) =
    \sum_{K\in \mathbb{K}^{n\times n}} p_K \prod_{i=1}^{n} a^{(K_{ii})}_{i}
  \end{align}
  if $p_K$ is symmetric ($p_K = p_{K'}$ if $K'$ is the same as $K$ up to a row and column permutation).
\end{theorem}

\noindent Let $A$ be a $2 \times 2$ matrix with eigenvalues $a_1,a_2$, then
\medskip \\
\indent \(
  a_1^n a_2^k + a_1^k a_2^n = \det(A^n_1, A^k_2) +\det(A^k_1, A^n_2)
\)
\medskip \\ \medskip
where $A^n_i$ is the $i$-th column of $A^n$.



\end{document}


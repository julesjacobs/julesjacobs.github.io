% \documentclass[a4paper, 11pt]{article}
\documentclass[a4paper, 11pt]{article}
\usepackage[scaled=0.84]{beramono}
\usepackage[dvipsnames]{xcolor}
\usepackage{titlesec}

\usepackage[a4paper,left=2.5cm,right=2.5cm,top=2.5cm,bottom=2.5cm]{geometry}
\usepackage[utf8]{inputenc}
\usepackage[T1]{fontenc}
\usepackage{xspace}
\usepackage[bitstream-charter]{mathdesign}
\usepackage{parskip}
\usepackage{microtype}
\usepackage{mathtools}
\usepackage{hyperref}
\usepackage{tikz}

\newcommand{\linkcolor}{Maroon}

\usepackage[all]{hypcap}
\hypersetup{
    colorlinks=true,
    linkcolor=\linkcolor,
    citecolor=\linkcolor,
    filecolor=\linkcolor,
    urlcolor=\linkcolor,
}
\usepackage[nameinlink,noabbrev]{cleveref}
\usepackage{float}
\usepackage{listings}
% \usepackage{tcolorbox}

\lstdefinelanguage{Python}{
  keywords={from, import, def, for, in, if, return, lambda},
  comment=[l]{\#},
  morestring=[b]',
  morestring=[b]",
}

\lstset{
   columns=fullflexible,
   language=Python,
  %  keywordstyle=\color{RoyalBlue!30!black},
  %  keywordstyle=\color{Maroon},
   keywordstyle=\bfseries\color{Maroon!65},
   commentstyle=\color{Green!80!black}\ttfamily,
   identifierstyle=\color{RoyalBlue!70!black},
   stringstyle=\color{red!50!black}\ttfamily,
   showstringspaces=false,
   extendedchars=true,
   basicstyle=\small\ttfamily,
   % breaklines=true,
}

\usepackage{amsmath}
\usepackage{amsthm}
\newtheorem{theorem}{Theorem}[section]
\newtheorem{lemma}[theorem]{Lemma}
\theoremstyle{definition}
\newtheorem{definition}{Definition}[section]
\newtheorem{corollary}{Corollary}[theorem]

\usepackage[shortlabels]{enumitem}
\setitemize{noitemsep, topsep=1pt, leftmargin=*}
\setenumerate{noitemsep, topsep=1pt, leftmargin=*}
% \setdescription{noitemsep, topsep=0pt, leftmargin=*}
\setdescription{itemsep=.5pt, topsep=0pt, leftmargin=8pt}

\usepackage{mathpartir}
\usepackage{todonotes}


% \usepackage{adjustbox}
% \usepackage[notref,notcite]{showkeys}
% \usepackage{multicol}


\newcommand{\N}{\mathbb{N}}
\newcommand{\Z}{\mathbb{Z}}
\newcommand{\Q}{\mathbb{Q}}
\newcommand{\R}{\mathbb{R}}
\newcommand{\C}{\mathbb{C}}
\newcommand{\ie}{\textit{i.e.,}\xspace}

\makeatletter
\renewcommand\maketitle
  {
  \begin{center}
   {\color{black}\Large{\sc\@title}}\\\bigskip%
   \large{\@author}\\\medskip%
   \small{\@date}\bigskip%
  \end{center}%
  }


% \newcommand{\sectionnumbercolor}{Maroon!40}
\newcommand{\sectionnumbercolor}{black!40}

\titleformat{\section}
  {\normalfont\fontsize{12}{15}\bfseries}
  {\llap{\color{\sectionnumbercolor}\thesection\quad}}
  {0em}
  {}

\titleformat{\subsection}
  {\normalfont\fontsize{11}{10}\bfseries}
  {\llap{\color{\sectionnumbercolor}\thesubsection\quad}}
  {0em}
  {}

\titleformat{\subsubsection}
  {\normalfont\fontsize{10}{10}\bfseries}
  {\llap{\color{\sectionnumbercolor}\thesubsubsection\quad}}
  {0em}
  {}

\newcommand{\equationnumbercolor}{black!50}
\newtagform{brackets}{\color{\equationnumbercolor}(}{)}
\usetagform{brackets}

\title{Arithmetic on Church numerals using a notational trick}
\author{Jules Jacobs}
\begin{document}
\maketitle

\begin{abstract}
  Church numerals allow us to represent numbers in pure lambda calculus. In this short note we'll see how to define addition, multiplication, and exponentiation on Church numerals using a cute notational trick. As a bonus, we'll see how to define predecessor and fast growing functions.
\end{abstract}

\newcommand{\ch}[1]{\mathbf{#1}}

\section{Addition, multiplication, and exponentiation}

Church repesents a natural number $n$ as a higher order function, which I'll denote $\ch{n}$.
The function $\ch{n}$ takes another function $f$ and composes $f$ with itself $n$ times:
\[
  \ch{n}\ f = \underbrace{f \circ f \cdots \circ f}_{n\text{ times}} = f^n
\]
We can convert a Church numeral $\ch{n}$ back to an ordinary nat by applying it to the ordinary successor function $S : \N \to \N$ given by $S\ n = n+1$:
then $\ch{n}\ S\ 0$ gives us back an ordinary natural number $n$ because $\ch{n}\ S\ 0$ is the $n$-fold application of the successor function to the number $0$, which just increments it $n$ times.

The first few Church numerals are:
\begin{align*}
  \ch{0} &\triangleq{} \lambda f.\lambda z. z\\
  \ch{1} &\triangleq{} \lambda f.\lambda z. f z\\
  \ch{2} &\triangleq{} \lambda f.\lambda z.f (f z)\\
  \ch{3} &\triangleq{} \lambda f.\lambda z.f (f (f z))
\end{align*}

Many descriptions of Church numerals will view them in that way: as a function that takes \emph{two} arguments $f$ and $z$ that computes $f (f (\dots (f z) \dots))$, but this point of view gets incredibly confusing when you try to define arithmetic on them, particularly multiplication and exponentiation.
So think about $\ch{n}\ f = f^n$ as performing $n$-fold function composition.

If will be helpful to introduce an alternative notation for function application:
\begin{align*}
  x^f \equiv f(x)
\end{align*}
This may seem strange, but using this notation we can \emph{define} the first few Church numerals as:
\begin{align*}
  f^\ch{0} &\triangleq{} \mathsf{id}\\
  f^\ch{1} &\triangleq{} f \\
  f^\ch{2} &\triangleq{} f \circ f\\
  f^\ch{3} &\triangleq{} f \circ f \circ f
\end{align*}
Note that on the left hand side, we are really defining $\ch{3}$ as the function $\ch{3}(f) \triangleq{} f \circ f \circ f$.

The advantage of this notation is apparent when defining addition and multiplication on Church numerals:
\begin{align*}
  f^{\ch{n}+\ch{m}} \triangleq{} f^\ch{n} \circ f^\ch{m} &&
  f^{\ch{n}\cdot\ch{m}} \triangleq{} (f^\ch{n})^\ch{m}
\end{align*}
Exponentiation of Church numerals is even better: our notation already makes $\ch{n}^\ch{m}$ do the right thing:
\begin{align*}
  \ch{n}^\ch{m} \equiv \ch{m}(\ch{n}) \tag{already does the right thing!}
\end{align*}

The notation makes the proofs that this does arithmetic correctly a triviality: if $[n]$ is the Church numeral corresponding to an ordinary natural number $n \in \N$, then
\begin{align*}
  f^{[n]+[m]} = f^{[n]} \circ f^{[m]} = f^n \circ f^m = f^{n+m} = f^{[n+m]}
\end{align*}
where $f^{[m]} \equiv [m](f)$ according to our notation, and $f^n$ for ordinary natural number $n \in \N$ is $n$-fold function composition.
The proofs for multiplication and exponentiation are analogous.

\section{Predecessor}
Surprisingly, defining the predecessor on Church numerals is the most difficult. I think this solution is due to Curry.

We define the function $f : \N \times \N \to \N \times \N$:
\begin{align*}
  f((a,b)) &= (s(a),a)
\end{align*}
If we start with $(0,x)$ and keep applying $f$ we get the following sequence:
\begin{align*}
  (0,x) \to (1,0) \to (2,1) \to (3,2) \to (4,3) \to \cdots
\end{align*}
So
\begin{align*}
  f^n((0,x))_1 &= n\\
  f^n((0,x))_2 &= \begin{cases}
    x & \text{if } n=0 \\
    n-1 & \text{if } n > 0
  \end{cases}
\end{align*}
So we can define the predecessor function:
\begin{align*}
  p = \lambda n. n\ f\ (0,0)
\end{align*}
So that $p(0) = 0$ and $p(n) = n-1$ for $n>0$.

\subsection{Pairs}

We made use of pairs to define the predecessor, so to use pure lambda calculus we need to define pairs in terms of lambda. We represent a pair $(a,b)$ as:
\begin{align*}
  (a,b) &= \lambda f. f\ a\ b
\end{align*}
We can extract the components by passing in the function $f$:
\begin{align*}
  \mathsf{fst} &= \lambda x. x\ (\lambda a. \lambda b. a)\\
  \mathsf{snd} &= \lambda x. x\ (\lambda a. \lambda b. b)\\
\end{align*}

\subsection{Disjoint union}

Another way to define the predecessor is with disjoint unions. We take:
\begin{align*}
  \mathsf{inl}(a) &= \lambda f. \lambda g. f a \\
  \mathsf{inr}(a) &= \lambda f. \lambda g. g a
\end{align*}
Then we can define:
\begin{align*}
  f(\mathsf{inl}(a)) &= \mathsf{inr}(a)\\
  f(\mathsf{inr}(a)) &= \mathsf{inr}(s(a))
\end{align*}
We can do this pattern match on an inl/inr by calling it with the two branches as arguments:
\begin{align*}
  f(x) &= x\ (\lambda a. \mathsf{inr}(a))\ (\lambda a.\mathsf{inr}(s(a)))
\end{align*}
And we can define:
\begin{align*}
  p(n) = (n\ f\ \mathsf{inl}(0))\ (\lambda x. x)\ (\lambda x. x)
\end{align*}

\section{Fast growing functions}
Given any function $g : N \to N$ we can define a series of ever faster growing functions as follows:
\begin{align*}
  f_0(n) &= g(n)\\
  f_{k+1}(n) &= f^n_k(n)
\end{align*}
We can define this function using Church numerals:
\begin{align*}
  f_k = k\ (\lambda f. \lambda n. n f n)\ g
\end{align*}
If we take $g = s$ then,
\begin{align*}
  f_0(n) &= n+1\\
  f_1(n) &= 2n\\
  f_2(n) &= 2^n \cdot n
\end{align*}
The function $A(n) = f_n(n)$ grows pretty quickly. We can play the same game again, by putting $g = A$, obtaining a sequence:
\begin{align*}
  h_0(n) &= A(n)\\
  h_{k+1}(n) &= h^n_k(n)
\end{align*}
To get a feeling for how fast this grows, consider $h_1$:
\begin{align*}
  h_1(n) &= h_0^n(n)\\
         &= A(A(A(\dots A(A(n)))))\\
         &= A(A(A(\dots A(f_n(n)))))\\
         &= A(A(A(\dots f_{f_n(n)}(f_n(n)))))
\end{align*}
An expression like $h_3(3)$ gives us a relatively short lambda term that will normalise to a huge term. We might as well start with $g(n) = n^n$ since that's even easier to write using Church numerals:
\begin{align*}
  g &= \lambda a. a\ a\\
  A &= \lambda k. k\ (\lambda f. \lambda n. n f n)\ g\ k\\
  h &= \lambda k. k\ (\lambda f. \lambda n. n f n)\ A\ k\\
  3 &= \lambda f. \lambda z. f (f (f\ z))\\
  X &= h\ 3
\end{align*}
You can't write down anything close to the number $X$ even if you were to write a hundred pages of towers of exponentials. Of course, we can continue this game, and define a sequence
\begin{align*}
  g_0 &= \lambda a. a\ a\\
  g_1 &= \lambda k. k\ (\lambda f. \lambda n. n f n)\ g_0\ k\\
  g_2 &= \lambda k. k\ (\lambda f. \lambda n. n f n)\ g_1\ k\\
  \dots
\end{align*}
Which can be generalised as:
\begin{align*}
  f(g) &= \lambda k. k\ (\lambda f. \lambda n. n f n)\ g\ k\\
  g_n &= f^n(g_0)
\end{align*}
So we get an even more compact, yet much larger number with:
\begin{align*}
  f &= \lambda g. \lambda k. k\ (\lambda f. \lambda n. n f n)\ g\ k\\
  Y &= (3\ f)\ (\lambda a. a a)\ 3
\end{align*}
Of course, you can easily define much faster growing functions. But here's a challenge: what's the shortest lambda term that normalises, but takes more than the age of the universe to normalise? Or: what's the largest Church numeral you can write down in less than 30 symbols?

Please let me know of any mistakes. I haven't checked for mistakes :)

--- Jules

\end{document}
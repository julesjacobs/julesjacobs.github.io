\documentclass[a4paper, 11pt]{article}
\usepackage[a4paper,left=2.5cm,right=2.5cm,top=2.5cm,bottom=2.5cm]{geometry}
\usepackage[utf8]{inputenc}
\usepackage[T1]{fontenc}
\usepackage[bitstream-charter]{mathdesign}
\usepackage{microtype}
\usepackage{textcomp}

\usepackage{mathtools}
\newcommand\SetSymbol[1][]{\nonscript\:#1\vert\allowbreak\nonscript\:\mathopen{}}
\providecommand\given{} % to make it exist
\DeclarePairedDelimiterX\Set[1]\{\}{\renewcommand\given{\SetSymbol[\delimsize]}#1}

\usepackage{hyperref}
\hypersetup{
    colorlinks=true,
    linkcolor=blue,
    filecolor=magenta,
    urlcolor=blue,
}
\usepackage{listings}
\usepackage{color}
\definecolor{lightgray}{rgb}{.9,.9,.9}
\definecolor{darkgray}{rgb}{.4,.4,.4}
\definecolor{purple}{rgb}{0.65, 0.12, 0.82}

\lstdefinelanguage{JavaScript}{
  keywords={typeof, new, true, false, catch, function, return, null, catch, switch, var, in, while, do, else, case, break, val, then, Inductive, Fixpoint, match, end},
  keywordstyle=\color{blue},
  ndkeywords={class, export, boolean, throw, implements, import, this},
  ndkeywordstyle=\color{darkgray}\bfseries,
  identifierstyle=\color{black},
  sensitive=false,
  comment=[l]{//},
  morecomment=[s]{/*}{*/},
  commentstyle=\color{purple}\ttfamily,
  stringstyle=\color{red}\ttfamily
}

\lstset{
   language=JavaScript,
   extendedchars=true,
   basicstyle=\small\ttfamily,
   showstringspaces=false,
   showspaces=false,
   literate=
    {forall}{$\forall$}1
    {->}{$\to$}1
    {=>}{$\implies$}1
    {fun}{$\lambda$}1
}

\newcommand{\N}{\mathbb{N}}
\newcommand{\Z}{\mathbb{Z}}
\newcommand{\Q}{\mathbb{Q}}
\newcommand{\R}{\mathbb{R}}

\usepackage{amsmath}
\usepackage{amsthm}

\usepackage[shortlabels]{enumitem}
\usepackage{mathpartir}
\newcommand{\mif}{\mathsf{if}\ }
\newcommand{\mthen}{\ \mathsf{then}\ }
\newcommand{\melse}{\ \mathsf{else}\ }
\newtheorem{theorem}{Theorem}[section]
\newtheorem{lemma}[theorem]{Lemma}
\theoremstyle{definition}
\newtheorem{definition}{Definition}[section]
\newtheorem{corollary}{Corollary}[theorem]
\usepackage{parskip}

\title{Noether's theorem: the simplest case}
\author{Jules Jacobs}
\begin{document}
\maketitle

Noether's theorem states that conservation laws correspond to symmetries:
\begin{itemize}
  \item Translation invariance $\iff$ conservation of momuntum
  \item Rotation invariance $\iff$ convervation of angular momentum
  \item Time translation invariance $\iff$ conservation of energy
\end{itemize}
Versions of this correspondence hold very generally throughout physics: in classical mechanics, field theory, relativity, quantum mechanics. In this note I'll explain the simplest case: the classical mechanics of a single particle in a potential $V(x,y)$ in two dimensions.
\medskip

Newton's laws $ma = F$ describe the motion of this particle:
\begin{align*}
  m \ddot x &= \frac{\partial V}{\partial x} &
  m \ddot y &= \frac{\partial V}{\partial y}
\end{align*}

\textbf{We see: if $\frac{\partial V}{\partial x} = 0$, then $m \ddot x = 0$ so the momentum $m \dot x$ is conserved.}

The condition on the function $V$ can be stated in different ways:
\begin{itemize}
  \item $\frac{\partial V}{\partial x} = 0$
  \item $V(x + \Delta x, y) = V(x,y)$ for all $\Delta x$
  \item $V(x,y)$ is independent of $x$
  \item $V(x,y) = f(y)$ for some function $f$
\end{itemize}

We say that \textbf{$V$ is invariant under translation in the $x$ direction}, and this \textbf{symmetry} of $V$ is connected with the \textbf{conservation law} $L(x,\dot x, y, \dot y) = m\dot x = \text{constant}$.

Symmetries of $V$ are not always so obvious. For instance, for a particle in a gravitational well,
\[
  V(x,y) = -\frac{G M m}{\sqrt{x^2 + y^2}}
\]
This potential depends on both variables $x,y$ but it has rotational symmetry. If we write this in polar coordinates $(r,\theta)$,
\[
  V(r,\theta) = -\frac{G M m}{r}
\]
and then $V(r,\theta)$ is independent of $\theta$. The question arises: is this symmetry associated with any conservation law? For the moment, take my word for it that these are Newton's laws in polar coordinates:

\begin{align*}
  m \ddot r &= \frac{\partial V}{\partial r} &
  \frac{d}{dt}(m r^2 \dot \theta) &= \frac{\partial V}{\partial \theta}
\end{align*}

\textbf{We see: if $\frac{\partial V}{\partial \theta} = 0$, then $\frac{d}{dt}(m r^2 \dot \theta) = 0$, so $L(r,\dot r, \theta, \dot \theta) = m r^2 \dot \theta$ is conserved!}

If we change coordinates back to $x,y$ we have $m r^2 \dot \theta = m \dot y x - m \dot x y$, which is the usual expression for angular momentum. We can verify that this is conserved in the original coordinate system: take the derivative, and substitute the equations of motion whenever $\ddot x$ and $\ddot y$ appear, and simplify the resulting expression to $0$.

The question is: \textbf{why did Newton's laws have the form} $\frac{d}{dt}[L_1] = \frac{\partial V}{\partial \theta}$ and $\frac{d}{dt}[L_2] = \frac{\partial V}{\partial r}$ and do they have this form in any coordinate system? If they do, then whenever one of the partial derivatives of $V$ is zero, then we have a conservation law.

We can see that Newton's laws have this form in any coordinate system using Lagrangian mechanics.

Noethers idea is that you can describe the symmetries in a different way, so that you do not need to find a coordinate system in which $V$ is independent of one of the the coordinates.


The existence of a coordinate system $(\alpha, \beta)$ in which $\frac{\partial V}{\partial \alpha} = 0$ corresponds to a \textbf{continuous symmetry} of $V(x,y)$.

\begin{align*}
  A_\theta (x,y) = (x \cos \theta - y \sin \theta, x \sin \theta + y \cos \theta) \\
  \frac{A_\theta(x,y)}{d\theta} = (y,-x) \\
  V(A_\theta(x,y)) = V(x,y) \\
  DV(A_\theta(x,y))\cdot A'_\theta = 0 \\
  DV(x,y)\cdot A'_0 = 0 \\
  DV(x,y)\cdot A'_0 = 0 \\
  f(x,y) \ddot x + g(x,y) \ddot y =& 0 \\
  (f(x,y) \dot x + g(x,y) \dot y)' =&
  \dot f(x,y) \dot x + f(x,y) \ddot x + \dot g(x,y) \dot y + g(x,y) \ddot y \\
  =& \dot f(x,y) \dot x + \dot g(x,y) \dot y \\
  =&  (f_1(x,y) \dot x + f_2(x,y) \dot y) \dot x + (g_1(x,y) \dot x + g_2(x,y) \dot y) \dot y \\
\end{align*}

HAS TO BE A SYMMETRY OF THE LAGRANGIAN, NOT JUST OF V!

So also ok if it's a symmetry of the kinetic energy and V separately.
Not ok, things like $(x,y) \mapsto (x + \theta y, y)$.
But ok things like $(x,y) \mapsto (x + \theta, y)$.

\end{document}
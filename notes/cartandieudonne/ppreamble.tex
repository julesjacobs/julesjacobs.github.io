\documentclass[a4paper, 11pt]{scrartcl}
\usepackage[utf8]{inputenc}
\usepackage[T1]{fontenc}
\usepackage[scaled=0.84]{beramono}
\usepackage[dottedtoc]{classicthesis}
\usepackage[a4paper,left=2.5cm,right=2.5cm,top=2.5cm,bottom=2.5cm]{geometry}
% \usepackage{mathpazo}
% \usepackage{newpxmath}
% \usepackage[smallerops]{newpxmath}
% \usepackage{newpxtext} \useosf
% \usepackage{newpxtext}
% \usepackage[T1,small,euler-digits]{eulervm}
\usepackage{ceulerpx}

\newcommand{\sectionnumbercolor}{Maroon!60}
\titleformat{\section}
            {\large}
            {\llap{\color{\sectionnumbercolor}\thesection\quad}}
            {0pt}
            {\spacedlowsmallcaps}

\titleformat{\subsection}
            {\normalsize}
            {\llap{\color{\sectionnumbercolor}\thesubsection\quad}}
            {0pt}
            {\textit}

\pagestyle{plain}

\usepackage{parskip}

\usepackage{mathtools}
\usepackage{hyperref}
\usepackage[all]{hypcap}
\hypersetup{
    linktoc=section
}
% \urlstyle{same}
\usepackage[nameinlink,noabbrev]{cleveref}
\usepackage{float}
\usepackage{listings}
\usepackage[dvipsnames]{xcolor}
\usepackage{tcolorbox}

\lstdefinelanguage{Python}{
  keywords={from, import, def, for, in, if, return, lambda},
  % keywordstyle=\bfseries\color{Maroon!65},
  comment=[l]{\#},
  morestring=[b]',
  morestring=[b]",
}

\lstdefinelanguage{Scala}{
  keywords={def, class, object, extends, case, for, if, else, var, val, match, enum},
  keywordstyle=\color{Maroon!80},
  morekeywords=[2]{Set,List,Re,Re2,Alt,Alt2,Seq,Seq2,Star,Star2,Chr,Chr2,Emp,Eps,Tm,Var,Lam,App,Sem,LamS,AppS,TmS,Char,String,Int,Db,Hs,Ty,Base,Arrow,Syn},
  keywordstyle=[2]\color{violet!80},
  comment=[l]{//},morecomment=[n]{/*}{*/},
  morestring=[b]',
  morestring=[b]",
  literate=
    {cdot}{$\cdot$ }1
    {->}{$\to$ }1
}

\lstset{
   columns=fullflexible,
   language=Scala,
   keywordstyle=\color{Maroon!80},
   commentstyle=\color{Green!80!black}\ttfamily,
   identifierstyle=\color{RoyalBlue!70!black},
   stringstyle=\color{red!50!black}\ttfamily,
   showstringspaces=false,
   extendedchars=true,
   basicstyle=\small\ttfamily,
   xleftmargin=1em,
   % breaklines=true,
}


\newcommand{\N}{\mathbb{N}}
\newcommand{\Z}{\mathbb{Z}}
\newcommand{\Q}{\mathbb{Q}}
\newcommand{\R}{\mathbb{R}}
\newcommand{\C}{\mathbb{C}}

\usepackage{amsmath}
\let\openbox\relax
\usepackage{amsthm}
\newtheorem{theorem}{Theorem}[section]
\newtheorem{lemma}{Lemma}[section]
\theoremstyle{definition}
\newtheorem{definition}{Definition}[section]
\newtheorem{corollary}{Corollary}[theorem]

\usepackage[shortlabels]{enumitem}
\setitemize{noitemsep, topsep=1pt, leftmargin=*}
\setenumerate{noitemsep, topsep=1pt, leftmargin=*}
\setdescription{itemsep=.5pt, topsep=0pt, leftmargin=8pt}
\usepackage{mathpartir}


\usepackage{adjustbox}
% \usepackage[notref,notcite]{showkeys}
\usepackage{todonotes}
\usepackage{multicol}

% =========== TIKZ ===========
% \usepackage{graphicx}
% \usepackage{tikz}
% \usetikzlibrary{shapes.geometric, arrows}
% \usetikzlibrary{fit}
% \usetikzlibrary{svg.path}
% % \usetikzlibrary{graphdrawing}
% % \usetikzlibrary{graphdrawing.force}
% % \usetikzlibrary{graphdrawing.layered}
% \usetikzlibrary{decorations}
% \usetikzlibrary{decorations.markings}
% \usetikzlibrary{backgrounds}


\newcommand\blfootnote[1]{%
  \begingroup
  \renewcommand\thefootnote{}\footnote{#1}%
  \addtocounter{footnote}{-1}%
  \endgroup
}
\newcommand{\ie}{\textit{i.e.,}\xspace}

\makeatletter
\renewcommand\maketitle
  {
  \begin{center}
   {\color{black}\large\spacedallcaps{\@title}}\\\bigskip%
   \large{\@author}\\\medskip%
   \normalsize{\@date}\bigskip%
  \end{center}%
  }

\newcommand{\equationnumbercolor}{black!60}
\newtagform{brackets}{\color{\equationnumbercolor}(}{)}
\usetagform{brackets}
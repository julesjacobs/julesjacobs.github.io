\documentclass[a4paper, 11pt]{scrartcl}
\usepackage[utf8]{inputenc}
\usepackage[scaled=0.88]{beraserif}
\usepackage[scaled=0.85]{berasans}
\usepackage[scaled=0.84]{beramono}
\usepackage[dottedtoc]{classicthesis}
% \clearscrheadfoot
% \ohead[]{\headmark}
% \cfoot[\pagemark]{\pagemark}
\usepackage[a4paper,left=2.5cm,right=2.5cm,top=2.5cm,bottom=2.5cm]{geometry}
\usepackage[T1]{fontenc}
% \usepackage{mathpazo}
% \linespread{1.05}
% \usepackage[T1,small,euler-digits]{eulervm}
% \usepackage{newpxmath}
\usepackage[smallerops]{newpxmath}
% \usepackage{newpxtext}
% \usepackage{eulerpx}

\setkomafont{disposition}{}
\setkomafont{section}{}
\newcommand{\sectioncolor}{black}
\newcommand{\sectionnumbercolor}{Maroon!60}
\titleformat{\section}
            {\color{\sectioncolor}\large\usekomafont{disposition}\usekomafont{section}}
            {\color{\sectionnumbercolor}\llap{\textsc{\MakeTextLowercase{\thesection}}\hspace{0.7em}}}
            {0pt}
            {\usekomafont{disposition}\usekomafont{section}\spacedlowsmallcaps}
            % [{\color{Maroon}\titlerule}]

\titleformat{\subsection}
            {\color{\sectioncolor}\large\usekomafont{disposition}\usekomafont{section}}
            {\color{\sectionnumbercolor}\llap{\textsc{\MakeTextLowercase{\thesubsection}}\hspace{0.7em}}}
            {0pt}
            {\usekomafont{disposition}\usekomafont{section}\spacedlowsmallcaps}

\pagestyle{plain}

% \usepackage[T1]{fontenc}
% \usepackage[bitstream-charter]{mathdesign}

% \RequirePackage[T1]{fontenc}
% \RequirePackage[tt=false, type1=true]{libertine}
% \RequirePackage[varqu]{zi4}
% \RequirePackage[libertine]{newtxmath}

\usepackage{xspace}
\usepackage{parskip}
\usepackage{microtype}
\usepackage{textcomp}
\usepackage{mathtools}
\usepackage{hyperref}
\usepackage[all]{hypcap}
\hypersetup{
    % colorlinks=true,
    % linkcolor=blue,
    % filecolor=magenta,
    % urlcolor=blue,
    linktoc=all
}
% \urlstyle{same}
\usepackage[nameinlink,noabbrev]{cleveref}
\usepackage{float}
\usepackage{listings}
\usepackage{color}
\usepackage[dvipsnames]{xcolor}
\usepackage{tcolorbox}

\lstdefinelanguage{Python}{
  keywords={from, import, def, for, in, if, return, lambda},
  comment=[l]{\#},
  morestring=[b]',
  morestring=[b]",
}

\lstset{
   columns=fullflexible,
   language=Python,
  %  keywordstyle=\color{RoyalBlue!30!black},
  %  keywordstyle=\color{Maroon},
   keywordstyle=\bfseries\color{Maroon!65},
   commentstyle=\color{Green!80!black}\ttfamily,
   identifierstyle=\color{RoyalBlue!70!black},
   stringstyle=\color{red!50!black}\ttfamily,
   showstringspaces=false,
   extendedchars=true,
   basicstyle=\small\ttfamily,
   % breaklines=true,
}


\newcommand{\N}{\mathbb{N}}
\newcommand{\Z}{\mathbb{Z}}
\newcommand{\Q}{\mathbb{Q}}
\newcommand{\R}{\mathbb{R}}
\newcommand{\C}{\mathbb{C}}

\usepackage{amsmath}
\let\openbox\undefined
\usepackage{amsthm}

\usepackage[shortlabels]{enumitem}
\setitemize{noitemsep, topsep=1pt, leftmargin=*}
\setenumerate{noitemsep, topsep=1pt, leftmargin=*}
% \setdescription{noitemsep, topsep=0pt, leftmargin=*}
\setdescription{itemsep=.5pt, topsep=0pt, leftmargin=8pt}
\usepackage{mathpartir}

\newtheorem{theorem}{Theorem}[section]
\newtheorem{lemma}[theorem]{Lemma}
\theoremstyle{definition}
\newtheorem{definition}{Definition}[section]
\newtheorem{corollary}{Corollary}[theorem]
\usepackage{adjustbox}
% \usepackage[notref,notcite]{showkeys}
\usepackage{todonotes}
\usepackage{multicol}

% =========== TIKZ ===========
% \usepackage{graphicx}
% \usepackage{tikz}
% \usetikzlibrary{shapes.geometric, arrows}
% \usetikzlibrary{fit}
% \usetikzlibrary{svg.path}
% % \usetikzlibrary{graphdrawing}
% % \usetikzlibrary{graphdrawing.force}
% % \usetikzlibrary{graphdrawing.layered}
% \usetikzlibrary{decorations}
% \usetikzlibrary{decorations.markings}
% \usetikzlibrary{backgrounds}
\newcommand\blfootnote[1]{%
  \begingroup
  \renewcommand\thefootnote{}\footnote{#1}%
  \addtocounter{footnote}{-1}%
  \endgroup
}
\newcommand{\ie}{\textit{i.e.,}\xspace}

\makeatletter
\renewcommand\maketitle
  {
  \begin{center}
   {\color{black}\large\spacedallcaps{\@title}}\\\bigskip%
   \large{\@author}\\\medskip%
   \normalsize{\@date}\bigskip%
  \end{center}%
  }

\newcommand{\equationnumbercolor}{black!60}
\newtagform{brackets}{\color{\equationnumbercolor}(}{)}
\usetagform{brackets}
\newcommand{\id}[1]{\lstinline\|#1\|}

\title{A proof of the Cartan-Dieudonné theorem}
\author{\large{Jules Jacobs}}
\date{\normalsize	\today}

\DeclareMathOperator{\im}{im}

\begin{document}
\maketitle
% \blfootnote{Contact information: \url{https://julesjacobs.com}}

\begin{abstract}
  \noindent
  The Cartan-Dieudonné theorem is fundamental theorem about the geometry of $n$-dimensional space:
  any orthogonal transformation $A$ can be written as a sequence of at most $n$ reflections.
  The proofs that I could find go by induction on $n$ and hence have to relate maps on $n-1$ dimensional spaces to maps on $n$ dimensional spaces.
  This leads to technicalities or handwaving.
  We'll see a slightly modified proof that stays in $n$ dimensions by doing induction on $\dim \ker(A - I)$.
\end{abstract}

\section{The Cartan-Dieudonné theorem}

\newcommand{\V}{\R^n}

The theorem we want to prove is:
\begin{theorem}[Cartan-Dieudonné]
  \label{thm:cd}
  An orthogonal transformation $A : \V \to \V$ can be written as a sequence of $k \leq n$ reflections in vectors $v_1,v_2,\dots,v_k \in \V$:
  \[ A = R_{v_1} R_{v_2} \cdots R_{v_k} \]
  where $k = n - \dim \ker (A - I)$.
\end{theorem}

The space $\ker(A - I) = \{ v \in \V \mid Av = v \}$ is the subspace where the transformation $A$ is the identity.
So the Cartan-Dieudonné theorem usually decomposes an orthogonal tranformation into $n$ reflections, but we save one reflection per direction where $A$ is the identity. We shall see that this is the minimum number: it cannot be done with even fewer reflections.

The idea of the proof is that if we have a vector $u$ such that $Au \neq u$, then we can compose $A$ with the refection $R_{(Au - u)}$, which sends $Au$ back to $u$, in order to make $A$ also the identity in that direction.
This reflection does not disturb any of the directions where $A$ was already the identity.
We prove the Cartan-Dieudonné theorem by iterating this processes until $A$ is the identity in all directions.
We shall now investigate this in more detail.

\section{The geometry of orthogonal tranformations}

\newcommand{\inner}[1]{\left<#1\right>}
\newcommand{\len}[1]{||#1||}

A linear map $A : \V \to \V$ is an orthogonal transformation if one of the following equivalent conditions holds:
\begin{enumerate}
  \item $A^T A = I$ (or equivalently, $A^T = A^{-1}$).
  \item $\inner{Av,Aw} = \inner{v,w}$ for all $v,w$.
  \item $\len{Av} = \len{v}$ for all $v$.
\end{enumerate}

Examples of orthogonal transformations are rotations and reflections.

The reflection $R_v : \V \to \V$ in a vector $v$ is defined as follows:
\begin{definition}
  $R_v \triangleq{} I - 2\frac{v v^T}{|v|^2}$
\end{definition}
On $\R^3$ for instance, $R_{(1,0,0)}(x,y,z) = (-x,y,z)$.

A reflection $R_v$ is the identity on a subspace of dimension $n - 1$ (namely the plane orthogonal to $v$), and really does something on a subspace of dimension 1.
Similarly, a rotation is the identity on a subspace of dimension $n - 2$ and really does something on a subspace of dimension $2$.
Note that the phrase "really does something" must be interpreted with care: a rotation moves almost all points of $\R^3$; only the axis of rotation is left fixed. But still, because these are linear maps, the action can be decomposed into a plane of rotation, and the identity on the subspace orthogonal to the plane.

The subspace on which $A$ is the identity is $\ker(A - I) = \{ u \mid A u = u \}$.
The subspace orthogonal to $\ker(A - I)$ on which $A$ really does something can be characterized in two equivalent ways:

\begin{lemma}
  \label{lem:imker}
  If $A$ is an orthogonal transformation, then $\ker(A - I)^\perp = \im(A - I)$.
\end{lemma}
\begin{proof}
  $\ker(A - I)^\perp = \im((A - I)^T) = \im(A^T - I) = \im(A^{-1} - I) = \im(A - I)$.
\end{proof}

This space is important because its dimension determines how many reflections we need when decomposing $A$: for directions in which $A$ already is the identity we don't need any reflections.

The idea behind the proof of the Cartan-Dieudonné theorem is that we can make $A$ be the identity in more directions by composing it with reflections, and repeat this until it is the identity in all directions.
This is given in the following lemma:

\begin{lemma}
  If $A$ is an orthogonal tranformation that is the identity in $k$ directions, then $R_v A$ is the identity in $k+1$ directions, where $v$ is \textbf{any} nonzero vector $v \in \im(A - I) \setminus \{ 0 \}$ (i.e., in the subspace where $A$ really does something).
\end{lemma}
\begin{proof}
  Let $v \in \im(A-I) \setminus \{ 0 \}$, so there is $u$ such that $v = Au - u \neq 0$.
  Then (1) $R_v A$ is still the identity everywhere $A$ is the identity (i.e., on $\ker(A - I)$), and (2) additionally $R_v A$ is also the identity on $u \notin \ker(A - I)$.

  To show (1), note that $R_v$ is the identity on all directions orthogonal to $v$, which by \Cref{lem:imker} includes everything in $\ker(A - I)$.

  To show (2), we can do an explicit calculation to show $R_{(Au - u)}Au = u$, but a picture is more instructive.
\end{proof}

To prove \Cref{thm:cd}, we can repeatedly apply this lemma until $A$ is the identity in all directions, so that we have $R_{v_k} \cdots R_{v_2} R_{v_1} A = I$, which gives $A = R_{v_1} R_{v_2} \cdots R_{v_k}$.

This is also the minimum number of reflections: if $A$ can be written as $k \leq n$ reflections, then there are at least $n - k$ directions where $A$ is the identity (e.g. if $A$ can be written as one reflection, then it is the identity in $n - 1$ directions).
The only directions in which $R_{v_1} R_{v_2} \cdots R_{v_k}$ is potentially not the identity is $span \{v_1, v_2, \cdots, v_k\}$.

\section{Todos}

\begin{lemma}
  If $\len{u} = \len{v}$ then $R_{(u - v)}u = v$, and if $w$ is orthogonal to $u,v$ then $R_{(u-v)}w = w$.
\end{lemma}

\begin{itemize}
  \item Restructure some stuff above; not very happy with the current text.
  \item Stuff about rotations and isometries.
  \item Stuff about QR decomposition.
\end{itemize}

% \section{Some consequences of the Cartan-Dieudonné theorem}

% \tableofcontents

% \newpage
% \bibliographystyle{alphaurl}
% \bibliography{references}


\end{document}
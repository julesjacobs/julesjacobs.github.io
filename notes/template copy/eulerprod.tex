
\documentclass[a4paper, 11pt]{article}
\usepackage[a4paper,left=2.5cm,right=2.5cm,top=2.5cm,bottom=2.5cm]{geometry}
\usepackage[utf8]{inputenc}
\usepackage[T1]{fontenc}
\usepackage{xspace}
\usepackage[bitstream-charter]{mathdesign}
\usepackage{parskip}
\usepackage{microtype}
\usepackage{textcomp}
\usepackage{mathtools}
\usepackage{hyperref}
\usepackage[all]{hypcap}
\hypersetup{
    colorlinks=true,
    linkcolor=blue,
    filecolor=magenta,
    urlcolor=blue,
}
\usepackage[nameinlink,noabbrev]{cleveref}
\usepackage{float}
\usepackage{listings}
\usepackage{color}
\usepackage[dvipsnames]{xcolor}
\usepackage{tcolorbox}
\lstdefinelanguage{JavaScript}{
  keywords={typeof, new, true, false, catch, function, return, null, catch, switch, var, while, do, else, case, break, val, then, Definition, Check, Lemma, Proof, Qed, Inductive, Fixpoint, match, for, class, object, extends, override, def, if},
  keywordstyle=\color{blue},
  comment=[l]{//},
  morecomment=[s]{/*}{*/},
  commentstyle=\color{purple}\ttfamily,
  stringstyle=\color{red}\ttfamily
}

\lstset{
   columns=fullflexible,
   language=JavaScript,
   extendedchars=true,
   basicstyle=\small\ttfamily,
   literate=
    % {epsilon}{$\epsilon$}1
    % {empty}{$\emptyset$}1
    % {forall}{$\forall$}1
    % {exists}{$\exists$}1
    % {<->}{$\iff$}1
    {->}{$\to\ $}1
    % {<-}{$\leftarrow\ $}1
    {=>}{$\implies\ $}1
    % {fun}{$\lambda$}1
    % {and}{$\wedge$}1
    % {or}{$\vee$}1
    {cdot}{$\cdot$ }1
}

\newcommand{\N}{\mathbb{N}}
\newcommand{\Z}{\mathbb{Z}}
\newcommand{\Q}{\mathbb{Q}}
\newcommand{\R}{\mathbb{R}}

\usepackage{amsmath}
\usepackage{amsthm}

\usepackage[shortlabels]{enumitem}
\setitemize{noitemsep, topsep=1pt, leftmargin=*}
\setenumerate{noitemsep, topsep=1pt, leftmargin=*}
% \setdescription{noitemsep, topsep=0pt, leftmargin=*}
\setdescription{itemsep=.5pt, topsep=0pt, leftmargin=8pt}
\usepackage{mathpartir}

\newtheorem{theorem}{Theorem}[section]
\newtheorem{lemma}[theorem]{Lemma}
\theoremstyle{definition}
\newtheorem{definition}{Definition}[section]
\newtheorem{corollary}{Corollary}[theorem]
\usepackage{adjustbox}
% \usepackage[notref,notcite]{showkeys}
\usepackage{todonotes}
\usepackage{multicol}

% =========== TIKZ ===========
% \usepackage{graphicx}
% \usepackage{tikz}
% \usetikzlibrary{shapes.geometric, arrows}
% \usetikzlibrary{fit}
% \usetikzlibrary{svg.path}
% % \usetikzlibrary{graphdrawing}
% % \usetikzlibrary{graphdrawing.force}
% % \usetikzlibrary{graphdrawing.layered}
% \usetikzlibrary{decorations}
% \usetikzlibrary{decorations.markings}
% \usetikzlibrary{backgrounds}

\newcommand{\ie}{\textit{i.e.,}\xspace}

\usepackage{lipsum}

\title{Euler product}
\author{Jules Jacobs}

\begin{document}
\maketitle

\begin{abstract}
\end{abstract}

Start with the formula for $df$ in terms of $dx$ \& $dy$:
\begin{align*}
  df = \frac{\partial f}{\partial x} dx
     + \frac{\partial f}{\partial y} dy
\end{align*}
This really means:
\begin{align*}
  df = \frac{\partial f}{\partial x}\bigg\rvert_y dx
     + \frac{\partial f}{\partial y}\bigg\rvert_x dy
\end{align*}
Wedge both sides with $dy$:
\begin{align*}
  df \wedge dy
  = \frac{\partial f}{\partial x}\bigg\rvert_y dx \wedge dy + \frac{\partial f}{\partial y}\bigg\rvert_x dy \wedge dy
  = \frac{\partial f}{\partial x}\bigg\rvert_y dx \wedge dy
\end{align*}
Therefore:
\begin{align*}
  \frac{\partial f}{\partial x}\bigg\rvert_y =
  \frac{df \wedge dy}{dx \wedge dy}
\end{align*}

Application: Euler product formula.
\begin{align*}
  \frac{\partial u}{\partial v}\bigg\rvert_w \cdot
  \frac{\partial v}{\partial w}\bigg\rvert_u \cdot
  \frac{\partial w}{\partial u}\bigg\rvert_v
&=\frac{du \wedge dw}{dv \wedge dw} \cdot
  \frac{dv \wedge du}{dw \wedge du} \cdot
  \frac{dw \wedge dv}{du \wedge dv} \\
&=\frac{du \wedge dw}{dw \wedge du} \cdot
  \frac{dv \wedge du}{du \wedge dv} \cdot
  \frac{dw \wedge dv}{dv \wedge dw} \\
&= (-1) \cdot (-1) \cdot (-1) \\
&= -1
\end{align*}

\newpage
Cramer's rule using the wedge product \cite{joot}.

Let $A$ be a $2 \times 2$ matrix and $b \in \R^2$. System of equations:

\begin{align*}
  A\begin{pmatrix}x\\y\end{pmatrix} = b
\end{align*}
That is,
\begin{align*}
  A_1 x + A_2 y = b
\end{align*}
Wedge both sides with $A_2$:
\begin{align*}
  A_1 \wedge A_2 x + A_2 \wedge A_2 y = b \wedge A_2
\end{align*}
So
\begin{align*}
  x = \frac{b \wedge A_2}{A_1 \wedge A_2}
\end{align*}

\begin{align*}
  b &= x A_1 + y A_2 & \text{$\implies$ wedge with $A_2$}\\
  df &= \frac{\partial f}{\partial x}\bigg\rvert_y dx
     + \frac{\partial f}{\partial y}\bigg\rvert_x dy& \text{$\implies$ wedge with $dy$}\\
\end{align*}

\newcommand{\pconst}[2]{\frac{\partial #1}{\partial #2}\bigg\rvert}

\begin{align*}
  \pconst{a}{b}_c \cdot \pconst{c}{d}_b \cdot \pconst{b}{a}_d \cdot \pconst{d}{c}_a = 1
\end{align*}

\bibliographystyle{alphaurl}
\bibliography{references}


\end{document}

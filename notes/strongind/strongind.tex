\documentclass[a4paper, 11pt]{scrartcl}
\usepackage[utf8]{inputenc}
\usepackage[scaled=0.88]{beraserif}
\usepackage[scaled=0.85]{berasans}
\usepackage[scaled=0.84]{beramono}
\usepackage[dottedtoc]{classicthesis}
% \clearscrheadfoot
% \ohead[]{\headmark}
% \cfoot[\pagemark]{\pagemark}
\usepackage[a4paper,left=2.5cm,right=2.5cm,top=2.5cm,bottom=2.5cm]{geometry}
\usepackage[T1]{fontenc}
% \usepackage{mathpazo}
% \linespread{1.05}
% \usepackage[T1,small,euler-digits]{eulervm}
% \usepackage{newpxmath}
\usepackage[smallerops]{newpxmath}
% \usepackage{newpxtext}
% \usepackage{eulerpx}

\setkomafont{disposition}{}
\setkomafont{section}{}
\newcommand{\sectioncolor}{black}
\newcommand{\sectionnumbercolor}{Maroon!60}
\titleformat{\section}
            {\color{\sectioncolor}\large\usekomafont{disposition}\usekomafont{section}}
            {\color{\sectionnumbercolor}\llap{\textsc{\MakeTextLowercase{\thesection}}\hspace{0.7em}}}
            {0pt}
            {\usekomafont{disposition}\usekomafont{section}\spacedlowsmallcaps}
            % [{\color{Maroon}\titlerule}]

\titleformat{\subsection}
            {\color{\sectioncolor}\large\usekomafont{disposition}\usekomafont{section}}
            {\color{\sectionnumbercolor}\llap{\textsc{\MakeTextLowercase{\thesubsection}}\hspace{0.7em}}}
            {0pt}
            {\usekomafont{disposition}\usekomafont{section}\spacedlowsmallcaps}

\pagestyle{plain}

% \usepackage[T1]{fontenc}
% \usepackage[bitstream-charter]{mathdesign}

% \RequirePackage[T1]{fontenc}
% \RequirePackage[tt=false, type1=true]{libertine}
% \RequirePackage[varqu]{zi4}
% \RequirePackage[libertine]{newtxmath}

\usepackage{xspace}
\usepackage{parskip}
\usepackage{microtype}
\usepackage{textcomp}
\usepackage{mathtools}
\usepackage{hyperref}
\usepackage[all]{hypcap}
\hypersetup{
    % colorlinks=true,
    % linkcolor=blue,
    % filecolor=magenta,
    % urlcolor=blue,
    linktoc=all
}
% \urlstyle{same}
\usepackage[nameinlink,noabbrev]{cleveref}
\usepackage{float}
\usepackage{listings}
\usepackage{color}
\usepackage[dvipsnames]{xcolor}
\usepackage{tcolorbox}

\lstdefinelanguage{Python}{
  keywords={from, import, def, for, in, if, return, lambda},
  comment=[l]{\#},
  morestring=[b]',
  morestring=[b]",
}

\lstset{
   columns=fullflexible,
   language=Python,
  %  keywordstyle=\color{RoyalBlue!30!black},
  %  keywordstyle=\color{Maroon},
   keywordstyle=\bfseries\color{Maroon!65},
   commentstyle=\color{Green!80!black}\ttfamily,
   identifierstyle=\color{RoyalBlue!70!black},
   stringstyle=\color{red!50!black}\ttfamily,
   showstringspaces=false,
   extendedchars=true,
   basicstyle=\small\ttfamily,
   % breaklines=true,
}


\newcommand{\N}{\mathbb{N}}
\newcommand{\Z}{\mathbb{Z}}
\newcommand{\Q}{\mathbb{Q}}
\newcommand{\R}{\mathbb{R}}
\newcommand{\C}{\mathbb{C}}

\usepackage{amsmath}
\let\openbox\undefined
\usepackage{amsthm}

\usepackage[shortlabels]{enumitem}
\setitemize{noitemsep, topsep=1pt, leftmargin=*}
\setenumerate{noitemsep, topsep=1pt, leftmargin=*}
% \setdescription{noitemsep, topsep=0pt, leftmargin=*}
\setdescription{itemsep=.5pt, topsep=0pt, leftmargin=8pt}
\usepackage{mathpartir}

\newtheorem{theorem}{Theorem}[section]
\newtheorem{lemma}[theorem]{Lemma}
\theoremstyle{definition}
\newtheorem{definition}{Definition}[section]
\newtheorem{corollary}{Corollary}[theorem]
\usepackage{adjustbox}
% \usepackage[notref,notcite]{showkeys}
\usepackage{todonotes}
\usepackage{multicol}

% =========== TIKZ ===========
% \usepackage{graphicx}
% \usepackage{tikz}
% \usetikzlibrary{shapes.geometric, arrows}
% \usetikzlibrary{fit}
% \usetikzlibrary{svg.path}
% % \usetikzlibrary{graphdrawing}
% % \usetikzlibrary{graphdrawing.force}
% % \usetikzlibrary{graphdrawing.layered}
% \usetikzlibrary{decorations}
% \usetikzlibrary{decorations.markings}
% \usetikzlibrary{backgrounds}
\newcommand\blfootnote[1]{%
  \begingroup
  \renewcommand\thefootnote{}\footnote{#1}%
  \addtocounter{footnote}{-1}%
  \endgroup
}
\newcommand{\ie}{\textit{i.e.,}\xspace}

\makeatletter
\renewcommand\maketitle
  {
  \begin{center}
   {\color{black}\large\spacedallcaps{\@title}}\\\bigskip%
   \large{\@author}\\\medskip%
   \normalsize{\@date}\bigskip%
  \end{center}%
  }

\newcommand{\equationnumbercolor}{black!60}
\newtagform{brackets}{\color{\equationnumbercolor}(}{)}
\usetagform{brackets}

\newcommand{\tac}[1]{\lstinline[mathescape]~#1~}
\newcommand{\ciff}{\ \leftrightarrow\ }
\newcommand{\hyp}{\tac{H}}
\newcommand{\hypB}{\tac{G}}
\newcommand{\var}{\tac{x}}
\newcommand{\varB}{\tac{y}}

\newtheorem*{nlemma}{Lemma}

\DeclareMathOperator*{\tr}{tr}
\DeclareMathOperator*{\lcm}{lcm}



\title{Coinduction by Induction}
\author{Jules Jacobs}

\begin{document}
\maketitle

\begin{enumerate}
  \item Weak induction and strong induction
  \item Knaster-Tarski and induction
  \item Two characterisations of the least and greatest fixed point (ordinals vs pre/postfixpoints)
\end{enumerate}

\section{Preliminaries}

\begin{definition}
  A \emph{poset} $P$ is a set with a \emph{partial order} $\leq$ that is
  \begin{itemize}
    \item Reflexive: $x \leq x$ for all $x \in P$
    \item Transitive: if $x \leq y$ and $y \leq z$ then $x \leq z$ for all $x,y,z \in P$
    \item Antisymmetric: if $x \leq y$ and $y \leq x$ then $x=y$ for all $x,y \in P$
  \end{itemize}

\end{definition}

\newcommand{\lb}[1]{\mathsf{lb}(#1)}
\newcommand{\ub}[1]{\mathsf{ub}(#1)}

\emph{Notation:} ``$x \leq A$'' where $x \in P$ and $A \subseteq P$ means ``$x \leq a$ for all $a \in A$'',
and similarly for $x \geq A$.

\begin{definition}
  A \emph{complete lattice} is a poset $P$ with meet and join operations on subsets $A \subseteq P$
  \begin{itemize}
    \item The meet $\bigwedge A \in P$ satisfies $\bigwedge A \leq A$, and for any other $x \leq A$, we have $\bigwedge A \geq x$
    \item The join $\bigvee A \in P$ satisfies $\bigvee A \geq A$, and for any other $x \geq A$, we have $\bigvee A \leq x$
  \end{itemize}
\end{definition}

\emph{Mnemonic:} $\bigwedge$ is like $\bigcap$/conjunction/minimum, and $\bigvee$ is like $\bigcup$/disjunction/maximum.

\begin{definition}
  A function $f : P \to Q$ is \emph{monotone} if $x \leq y$ implies $f(x) \leq f(y)$
\end{definition}

Now let $f : P \to P$ be monotone.
Take an element $x$ that decreases when we apply $f$, i.e., $f(x) \leq x$.
By monotonicity, $f(f(x)) \leq f(x)$, i.e., if we apply $f$ again, the point decreases again.
(Similarly, points that increase keep increasing.)

Since $1$ can't help but decrease, we have a decreasing sequence $1 \geq f(1) \geq f(f(1)) \geq \cdots$.
If this sequence ever stabilises, we have found a fixed point of $f$.
If it never stabilises, we can append the meet over the entire sequence onto the end of the sequence to get an even lower point,
and then continue iterating from there.
If we continue doing this and iterate over a sufficiently large ordinal,
we get a sequence with cardinality greater than $P$.
This decreasing sequence can't have all distinct values,
and thus the sequence must stabilise at a fixed point of $f$.

The fixed point must be the greatest fixed point of $f$,
because assume that there is some other fixed point $x$ of $f$.
We have $x \leq 1$ so this fixed point lies below the first element of our sequence.
By monotonicity $f(x) \leq f(1)$, and since $f(x) = x$, it also lies below the second element of our sequence.
By transfinite induction and the property of the meet, we can show that $x$ lies below our entire sequence, including the fixed point that the sequence converges to.
So our sequence converges the greaest fixed point.

With the preceding method, the intermediate elements $f^\alpha(0)$ lie above the fixed point.
There is an alternative way of constructing the greastest fixed point of $f$ that approaches it from below.
Let $I = \{ x \in P \mid f(x) \geq x \}$ be the increasing elements in the lattice.
I claim that the greatest element $\bigvee I$ is also the greatest fixed point of $f$.

Proof:
\begin{enumerate}
  \item Take an arbitrary $x \in I$. Then $x \leq \bigvee I$, so by monotonicity $f(x) \leq f(\bigvee I)$.
        But also $x \leq f(x)$, so $x \leq f(\bigvee I)$.
  \item $f(\bigvee I) \geq I$
  \item $f(\bigvee I) \geq \bigvee I$
  \item $f(\bigvee I) \in I$
  \item $f(\bigvee I) \leq \bigvee I$.
  \item $f(\bigvee I) = \bigvee I$.
  \item Every fixpoint of $f$ is in $I$, so $\bigvee I$ is the greatest fixpoint.
\end{enumerate}





\begin{theorem}[Knaster-Tarski]
  Every monotone $f : P \to P$ has a least and greatest fixed point.
\end{theorem}

\begin{align*}
  \mu f := \bigwedge \{ x \in P \mid f(x) \leq x \}\\
  \nu f := \bigvee \{ x \in P \mid f(x) \geq x \}
\end{align*}

Let $D = \{ x \in P \mid f(x) \leq x \}$ be the set points that decrease when we apply $f$.
We have
\begin{itemize}
  \item $f[D] \subseteq D$ by monotonicity of $f$
  \item $\mu f \leq D$ by the meet
  \item $\mu f \leq f(\mu f)$ by combining the above two
\end{itemize}
\end{document}
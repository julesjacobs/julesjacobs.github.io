\documentclass[a4paper, 11pt]{scrartcl}
\usepackage[utf8]{inputenc}
\usepackage[scaled=0.88]{beraserif}
\usepackage[scaled=0.85]{berasans}
\usepackage[scaled=0.84]{beramono}
\usepackage[dottedtoc]{classicthesis}
% \clearscrheadfoot
% \ohead[]{\headmark}
% \cfoot[\pagemark]{\pagemark}
\usepackage[a4paper,left=2.5cm,right=2.5cm,top=2.5cm,bottom=2.5cm]{geometry}
\usepackage[T1]{fontenc}
\usepackage{mathpazo}
\linespread{1.05}
\usepackage[T1,small,euler-digits]{eulervm}
\setkomafont{disposition}{}
\setkomafont{section}{}
\newcommand{\sectioncolor}{black}
\newcommand{\sectionnumbercolor}{Maroon!60}
\titleformat{\section}
            {\color{\sectioncolor}\large\usekomafont{disposition}\usekomafont{section}}
            {\color{\sectionnumbercolor}\llap{\textsc{\MakeTextLowercase{\thesection}}\hspace{0.7em}}}
            {0pt}
            {\usekomafont{disposition}\usekomafont{section}\spacedlowsmallcaps}
            % [{\color{Maroon}\titlerule}]

\titleformat{\subsection}
            {\color{\sectioncolor}\large\usekomafont{disposition}\usekomafont{section}}
            {\color{\sectionnumbercolor}\llap{\textsc{\MakeTextLowercase{\thesubsection}}\hspace{0.7em}}}
            {0pt}
            {\usekomafont{disposition}\usekomafont{section}\spacedlowsmallcaps}

\pagestyle{plain}

% \usepackage[T1]{fontenc}
% \usepackage[bitstream-charter]{mathdesign}

% \RequirePackage[T1]{fontenc}
% \RequirePackage[tt=false, type1=true]{libertine}
% \RequirePackage[varqu]{zi4}
% \RequirePackage[libertine]{newtxmath}

\usepackage{xspace}
\usepackage{parskip}
\usepackage{microtype}
\usepackage{textcomp}
\usepackage{mathtools}
\usepackage{hyperref}
\usepackage[all]{hypcap}
\hypersetup{
    % colorlinks=true,
    % linkcolor=blue,
    % filecolor=magenta,
    % urlcolor=blue,
    linktoc=all
}
\usepackage[nameinlink,noabbrev]{cleveref}
\usepackage{float}
\usepackage{listings}
\usepackage{color}
\usepackage[dvipsnames]{xcolor}
\usepackage{tcolorbox}

\lstdefinelanguage{Python}{
  keywords={from, import, def, for, in, if, return, lambda},
  comment=[l]{\#},
  morestring=[b]',
  morestring=[b]",
}

\lstset{
   columns=fullflexible,
   language=Python,
  %  keywordstyle=\color{RoyalBlue!30!black},
  %  keywordstyle=\color{Maroon},
   keywordstyle=\bfseries\color{Maroon!65},
   commentstyle=\color{Green!80!black}\ttfamily,
   identifierstyle=\color{RoyalBlue!70!black},
   stringstyle=\color{red!50!black}\ttfamily,
   showstringspaces=false,
   extendedchars=true,
   basicstyle=\small\ttfamily,
   % breaklines=true,
}


\newcommand{\N}{\mathbb{N}}
\newcommand{\Z}{\mathbb{Z}}
\newcommand{\Q}{\mathbb{Q}}
\newcommand{\R}{\mathbb{R}}
\newcommand{\C}{\mathbb{C}}

\usepackage{amsmath}
\let\openbox\undefined
\usepackage{amsthm}

\usepackage[shortlabels]{enumitem}
\setitemize{noitemsep, topsep=1pt, leftmargin=*}
\setenumerate{noitemsep, topsep=1pt, leftmargin=*}
% \setdescription{noitemsep, topsep=0pt, leftmargin=*}
\setdescription{itemsep=.5pt, topsep=0pt, leftmargin=8pt}
\usepackage{mathpartir}

\newtheorem{theorem}{Theorem}[section]
\newtheorem{lemma}[theorem]{Lemma}
\theoremstyle{definition}
\newtheorem{definition}{Definition}[section]
\newtheorem{corollary}{Corollary}[theorem]
\usepackage{adjustbox}
% \usepackage[notref,notcite]{showkeys}
\usepackage{todonotes}
\usepackage{multicol}

% =========== TIKZ ===========
% \usepackage{graphicx}
% \usepackage{tikz}
% \usetikzlibrary{shapes.geometric, arrows}
% \usetikzlibrary{fit}
% \usetikzlibrary{svg.path}
% % \usetikzlibrary{graphdrawing}
% % \usetikzlibrary{graphdrawing.force}
% % \usetikzlibrary{graphdrawing.layered}
% \usetikzlibrary{decorations}
% \usetikzlibrary{decorations.markings}
% \usetikzlibrary{backgrounds}
\newcommand\blfootnote[1]{%
  \begingroup
  \renewcommand\thefootnote{}\footnote{#1}%
  \addtocounter{footnote}{-1}%
  \endgroup
}
\newcommand{\ie}{\textit{i.e.,}\xspace}

\makeatletter
\renewcommand\maketitle
  {
  \begin{center}
   {\color{black}\large\spacedallcaps{\@title}}\\\bigskip%
   \large{\@author}\\\medskip%
   \normalsize{\@date}\bigskip%
  \end{center}%
  }

\newcommand{\tac}[1]{\lstinline[mathescape]~#1~}
\newcommand{\ciff}{\ \leftrightarrow\ }
\newcommand{\hyp}{\tac{H}}
\newcommand{\hypB}{\tac{G}}
\newcommand{\var}{\tac{x}}
\newcommand{\varB}{\tac{y}}

\newtheorem*{nlemma}{Lemma}

\DeclareMathOperator*{\tr}{tr}
\DeclareMathOperator*{\lcm}{lcm}



\title{The product of GCD and LCM}
\author{Jules Jacobs}

\begin{document}
\maketitle

This is the standard identity for the product of gcd and lcm:
\begin{align*}
  \gcd(a,b) \cdot \lcm(a,b) = ab
\end{align*}
One might wonder whether it holds that $\gcd(a,b,c) \cdot \lcm(a,b,c) = abc$.
Unfortunately, it does not; consider $a=b=c=2$. It does however hold that
\begin{align}
  \label{identity3}
  \gcd(a,b,c) \cdot \lcm(ab,ac,bc) = abc
\end{align}
In fact, it also holds that
\begin{align*}
  \gcd(ab,ac,bc) \cdot \lcm(a,b,c) = abc
\end{align*}
To see this, think of a number as a vector of its prime factorisation:
\begin{align*}
  2^2\cdot 3^1 \cdot 7^2 = (2,1,0,2,0,0,\cdots)
\end{align*}
On this representation, the gcd corresponds to taking the pointwise minimum, and the lcd the pointwise maximum:
\begin{align*}
  \gcd((a_1,a_2,\cdots),(b_1,b_2,\cdots),(c_1,c_2,c_3,\cdots)) &= (\min(a_1,b_1,c_1), \min(a_2,b_2,c_2), \cdots)\\
  \lcm((a_1,a_2,\cdots),(b_1,b_2,\cdots),(c_1,c_2,c_3,\cdots)) &= (\min(a_1,b_1,c_1), \max(a_2,b_2,c_2), \cdots)
\end{align*}
And the product corresponds to the pointwise sum:
\begin{align*}
  (a_1,a_2,\cdots)\cdot(b_1,b_2,\cdots)\cdot(c_1,c_2,c_3,\cdots) = (a_1+b_1+c_1, a_2+b_2+c_2, \cdots)
\end{align*}
Thus, in this representation, \cref{identity3} translates to:
\begin{align*}
  \min(a_i,b_i,c_i) + \max(a_i+b_i, a_i+c_i, b_i+c_i) = a_i + b_i + c_i \text{\qquad (for all i)}
\end{align*}
Now it is easy to see that the identity holds: fix $i$ and assume without loss of generality that $a_i \leq b_i \leq c_i$,
then the minimum reduces do $a_i$ and the maximum to $b_i+c_i$.

We see that more generally, given $n$ numbers instead of 3 numbers:
\begin{align*}
  \gcd(\text{k-fold products})\cdot \lcm(\text{(n-k)-fold products}) = \text{product}
\end{align*}
For $n=4$, this gives that the following values are all equal to $abcd$.
\begin{align*}
  gcd(\emptyset)&\cdot\lcm(abcd) &(k=0)\\
  gcd(a,b,c,d)&\cdot\lcm(bcd,acd,abd,abc) &(k=1)\\
  gcd(ab,ab,ad,bc,bd,cd)&\cdot\lcm(ab,ab,ad,bc,bd,cd) &(k=2)\\
  gcd(bcd,acd,abd,abc)&\cdot\lcm(a,b,c,d) &(k=3)\\
  gcd(abcd)&\cdot\lcm(\emptyset) &(k=4)
\end{align*}
In fact, if we allow negative powers in the prime factorization,
we can see that such identities hold over the positive rationals too,
with gcd and lcm suitably extended.

\end{document}
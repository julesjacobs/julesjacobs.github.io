\documentclass[a4paper, 11pt]{article}
\usepackage[a4paper,left=2.5cm,right=2.5cm,top=2.5cm,bottom=2.5cm]{geometry}
\usepackage[utf8]{inputenc}
\usepackage[T1]{fontenc}
\usepackage{xspace}
\usepackage[bitstream-charter]{mathdesign}
\usepackage{parskip}
\usepackage{microtype}
\usepackage{textcomp}
\usepackage{mathtools}
\usepackage{hyperref}
\usepackage[all]{hypcap}
\hypersetup{
    colorlinks=true,
    linkcolor=blue,
    filecolor=magenta,
    urlcolor=blue,
}
\usepackage[nameinlink,noabbrev]{cleveref}
\usepackage{float}
\usepackage{listings}
\usepackage{color}
\usepackage[dvipsnames]{xcolor}
\usepackage{tcolorbox}
\lstdefinelanguage{JavaScript}{
  keywords={typeof, new, true, false, catch, function, return, null, catch, switch, var, while, do, else, case, break, val, then, Definition, Check, Lemma, Proof, Qed, Inductive, Fixpoint, match, for, class, object, extends, override, def, if},
  keywordstyle=\color{blue},
  comment=[l]{//},
  morecomment=[s]{/*}{*/},
  commentstyle=\color{purple}\ttfamily,
  stringstyle=\color{red}\ttfamily
}

\lstset{
   columns=fullflexible,
   language=JavaScript,
   extendedchars=true,
   basicstyle=\small\ttfamily,
   literate=
    % {epsilon}{$\epsilon$}1
    % {empty}{$\emptyset$}1
    % {forall}{$\forall$}1
    % {exists}{$\exists$}1
    % {<->}{$\iff$}1
    {->}{$\to\ $}1
    % {<-}{$\leftarrow\ $}1
    {=>}{$\implies\ $}1
    % {fun}{$\lambda$}1
    % {and}{$\wedge$}1
    % {or}{$\vee$}1
    {cdot}{$\cdot$ }1
}

\newcommand{\N}{\mathbb{N}}
\newcommand{\Z}{\mathbb{Z}}
\newcommand{\Q}{\mathbb{Q}}
\newcommand{\R}{\mathbb{R}}

\usepackage{amsmath}
\usepackage{amsthm}

\usepackage[shortlabels]{enumitem}
\setitemize{noitemsep, topsep=1pt, leftmargin=*}
\setenumerate{noitemsep, topsep=1pt, leftmargin=*}
% \setdescription{noitemsep, topsep=0pt, leftmargin=*}
\setdescription{itemsep=.5pt, topsep=0pt, leftmargin=8pt}
\usepackage{mathpartir}

\newtheorem{theorem}{Theorem}
\newtheorem{lemma}[theorem]{Lemma}
\theoremstyle{definition}
\newtheorem{definition}{Definition}[section]
\newtheorem{corollary}{Corollary}[theorem]
\usepackage{adjustbox}
% \usepackage[notref,notcite]{showkeys}
\usepackage{todonotes}
\usepackage{multicol}

% =========== TIKZ ===========
% \usepackage{graphicx}
% \usepackage{tikz}
% \usetikzlibrary{shapes.geometric, arrows}
% \usetikzlibrary{fit}
% \usetikzlibrary{svg.path}
% % \usetikzlibrary{graphdrawing}
% % \usetikzlibrary{graphdrawing.force}
% % \usetikzlibrary{graphdrawing.layered}
% \usetikzlibrary{decorations}
% \usetikzlibrary{decorations.markings}
% \usetikzlibrary{backgrounds}

\newcommand{\ie}{\textit{i.e.,}\xspace}

\title{Bounded clause elimination}
\author{Jules Jacobs}

\begin{document}
\maketitle

Bounded variable elimination and blocked clause elimination are two effective SAT preprocessing techniques. This note is about forms of clause elimination that generalize both \cite{kiesl17}.

\newcommand{\elim}{\mathsf{elim}}

Given a CNF formula $F$ and a clause $c \in F$ and a literal $l \in c$, define $\elim(F,c,l)$ to be the CNF formula $F$ with clause $c$ \emph{replaced} by all resolvents of $c$ along $l$.

The formula F consists of clause $c$, clauses that contain $l$, clauses that contain $\neg l$, and clauses that contain neither $l$ nor $\neg l$:
\begin{align*}
  F = (l \vee \vec{c}) \wedge (\bigwedge_i l \vee \vec{a}_i) \wedge (\bigwedge_j \neg l \vee \vec{b}_j) \wedge (\bigwedge_k \vec{d}_k)
\end{align*}
Now $\elim(F,c,l)$ is:
\begin{align*}
  \elim(F,c,l) = (\bigwedge_j \vec{c} \vee \vec{b}_j) \wedge (\bigwedge_i l \vee \vec{a}_i) \wedge (\bigwedge_j \neg l \vee \vec{b}_j) \wedge (\bigwedge_k \vec{d}_k)
\end{align*}
It is clear that $F \implies \elim(F,c,l)$ because we've only added resolvents, but the reverse implication does not hold because we've deleted the clause $l \vee \vec{c}$. Take $F = l$, for example; then eliminating the only clause $l$ gives us the empty CNF, which is satisfied for any variable assignment, whereas $F$ is only satisfied for $l = 1$. However, the two formulas are equisatisfiable.

\begin{lemma}
  $F$ and $\elim(F,c,l)$ are equisatisfiable.
\end{lemma}
\begin{proof}
  Since $F \implies \elim(F,c,l)$, it suffices to show that any assignment for $\elim(F,c,l)$ can be turned into an assignment for $F$. If the clause $l \vee \vec{c}$ is satisfied by the assignment for $\elim(F,c,l)$, then we can use the same assignment to satisfy $F$, because the remaining clauses in $F$ are also in $\elim(F,c,l)$. So suppose $l = 0$ and $\vec{c} = 0$ in the assignment that satisfies $\elim(F,c,l)$. Then $\elim(F,c,l)$ simplifies to:
  \begin{align*}
    \elim(F,c,l) = (\bigwedge_j \vec{b}_j) \wedge (\bigwedge_i \vec{a}_i) \wedge (\bigwedge_k \vec{d}_k)
  \end{align*}
  Given this assignment for all variables except $l$, the formula $F$ simplifies to:
  \begin{align*}
    F = (l \vee \vec{c})
  \end{align*}
  Hence the same assignment but with $l = 1$ instead of $l = 0$ satisfies $F$.
\end{proof}

The proof of this lemma gives us a method to reconstruct solutions for $F$ from solutions for $\elim(F,c,l)$: if the clause we eliminated is already satisfied, do nothing, and otherwise flip the value of $l$.

We can do \emph{bounded clause elimination} by heuristically picking clauses to eliminate. We can simulate both blocked clause elimination and bounded variable elimination using $\elim$:

\begin{itemize}
  \item \textbf{Blocked clause elimination} deletes a clause $c$ if there is a literal $l \in c$ such that all resolvents of $c$ along $l$ are tautologies. This is equivalent to \emph{replacing} $c$ by the resolvents.
  \item \textbf{Bounded variable elimination} chooses a literal $l$ are replaces all clauses involving $l$ by all their resolvents. This is the same as running clause elimination multiple times, once for each clause that contains $l$.
\end{itemize}

\subsection*{Clause deletion}

A slightly different perspective is clause deletion: when is it safe to delete a clause? Deleting a clause may increase the number of satisfying assignments, but that is fine as long as (a) it doesn't turn an UNSAT problem into a SAT problem and (b) we have a method to reconstruct a satisfying assignment for the original problem from a satisfying assignment for the new problem.

The argument above shows that it is safe to delete a clause $c$ when all its resolvents along $l$ are implied by the remaining clauses. The solution reconstruction method is the same: if $c$ is not satisfied, flip $l$.

We can still simulate bounded variable elimination: first add all resolvents, and now we can delete the original clauses because all their resolvents are (trivially) implied.

\subsection*{Implementation in a solver}

\begin{itemize}
  \item Keep track of a stack of deleted clauses, and which literal $l$ was used to delete it.
  \item We can delete a clause at any time if its resolvents along some $l$ are implied by permanent clauses.
  \item Whenever the user adds a new clause containing $\neg l$, restore all clauses that were deleted using $l$. \\
  (Adding the assumption $l=0$ can be treated as adding the unit clause $\neg l$.)
  \item To reconstruct the original solution, pop all deleted clauses from the stack, flipping $l$ if necessary to make the clause satisfied.
\end{itemize}

\nocite{*}
\bibliographystyle{alphaurl}
\bibliography{references}


\end{document}
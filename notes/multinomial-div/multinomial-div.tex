\documentclass[a4paper, 11pt]{scrartcl}
\usepackage[utf8]{inputenc}
\usepackage[scaled=0.88]{beraserif}
\usepackage[scaled=0.85]{berasans}
\usepackage[scaled=0.84]{beramono}
\usepackage[dottedtoc]{classicthesis}
% \clearscrheadfoot
% \ohead[]{\headmark}
% \cfoot[\pagemark]{\pagemark}
\usepackage[a4paper,left=2.5cm,right=2.5cm,top=2.5cm,bottom=2.5cm]{geometry}
\usepackage[T1]{fontenc}
% \usepackage{mathpazo}
% \linespread{1.05}
% \usepackage[T1,small,euler-digits]{eulervm}
% \usepackage{newpxmath}
\usepackage[smallerops]{newpxmath}
% \usepackage{newpxtext}
% \usepackage{eulerpx}

\setkomafont{disposition}{}
\setkomafont{section}{}
\newcommand{\sectioncolor}{black}
\newcommand{\sectionnumbercolor}{Maroon!60}
\titleformat{\section}
            {\color{\sectioncolor}\large\usekomafont{disposition}\usekomafont{section}}
            {\color{\sectionnumbercolor}\llap{\textsc{\MakeTextLowercase{\thesection}}\hspace{0.7em}}}
            {0pt}
            {\usekomafont{disposition}\usekomafont{section}\spacedlowsmallcaps}
            % [{\color{Maroon}\titlerule}]

\titleformat{\subsection}
            {\color{\sectioncolor}\large\usekomafont{disposition}\usekomafont{section}}
            {\color{\sectionnumbercolor}\llap{\textsc{\MakeTextLowercase{\thesubsection}}\hspace{0.7em}}}
            {0pt}
            {\usekomafont{disposition}\usekomafont{section}\spacedlowsmallcaps}

\pagestyle{plain}

% \usepackage[T1]{fontenc}
% \usepackage[bitstream-charter]{mathdesign}

% \RequirePackage[T1]{fontenc}
% \RequirePackage[tt=false, type1=true]{libertine}
% \RequirePackage[varqu]{zi4}
% \RequirePackage[libertine]{newtxmath}

\usepackage{xspace}
\usepackage{parskip}
\usepackage{microtype}
\usepackage{textcomp}
\usepackage{mathtools}
\usepackage{hyperref}
\usepackage[all]{hypcap}
\hypersetup{
    % colorlinks=true,
    % linkcolor=blue,
    % filecolor=magenta,
    % urlcolor=blue,
    linktoc=all
}
% \urlstyle{same}
\usepackage[nameinlink,noabbrev]{cleveref}
\usepackage{float}
\usepackage{listings}
\usepackage{color}
\usepackage[dvipsnames]{xcolor}
\usepackage{tcolorbox}

\lstdefinelanguage{Python}{
  keywords={from, import, def, for, in, if, return, lambda},
  comment=[l]{\#},
  morestring=[b]',
  morestring=[b]",
}

\lstset{
   columns=fullflexible,
   language=Python,
  %  keywordstyle=\color{RoyalBlue!30!black},
  %  keywordstyle=\color{Maroon},
   keywordstyle=\bfseries\color{Maroon!65},
   commentstyle=\color{Green!80!black}\ttfamily,
   identifierstyle=\color{RoyalBlue!70!black},
   stringstyle=\color{red!50!black}\ttfamily,
   showstringspaces=false,
   extendedchars=true,
   basicstyle=\small\ttfamily,
   % breaklines=true,
}


\newcommand{\N}{\mathbb{N}}
\newcommand{\Z}{\mathbb{Z}}
\newcommand{\Q}{\mathbb{Q}}
\newcommand{\R}{\mathbb{R}}
\newcommand{\C}{\mathbb{C}}

\usepackage{amsmath}
\let\openbox\undefined
\usepackage{amsthm}

\usepackage[shortlabels]{enumitem}
\setitemize{noitemsep, topsep=1pt, leftmargin=*}
\setenumerate{noitemsep, topsep=1pt, leftmargin=*}
% \setdescription{noitemsep, topsep=0pt, leftmargin=*}
\setdescription{itemsep=.5pt, topsep=0pt, leftmargin=8pt}
\usepackage{mathpartir}

\newtheorem{theorem}{Theorem}[section]
\newtheorem{lemma}[theorem]{Lemma}
\theoremstyle{definition}
\newtheorem{definition}{Definition}[section]
\newtheorem{corollary}{Corollary}[theorem]
\usepackage{adjustbox}
% \usepackage[notref,notcite]{showkeys}
\usepackage{todonotes}
\usepackage{multicol}

% =========== TIKZ ===========
% \usepackage{graphicx}
% \usepackage{tikz}
% \usetikzlibrary{shapes.geometric, arrows}
% \usetikzlibrary{fit}
% \usetikzlibrary{svg.path}
% % \usetikzlibrary{graphdrawing}
% % \usetikzlibrary{graphdrawing.force}
% % \usetikzlibrary{graphdrawing.layered}
% \usetikzlibrary{decorations}
% \usetikzlibrary{decorations.markings}
% \usetikzlibrary{backgrounds}
\newcommand\blfootnote[1]{%
  \begingroup
  \renewcommand\thefootnote{}\footnote{#1}%
  \addtocounter{footnote}{-1}%
  \endgroup
}
\newcommand{\ie}{\textit{i.e.,}\xspace}

\makeatletter
\renewcommand\maketitle
  {
  \begin{center}
   {\color{black}\large\spacedallcaps{\@title}}\\\bigskip%
   \large{\@author}\\\medskip%
   \normalsize{\@date}\bigskip%
  \end{center}%
  }

\newcommand{\equationnumbercolor}{black!60}
\newtagform{brackets}{\color{\equationnumbercolor}(}{)}
\usetagform{brackets}

\newcommand{\tac}[1]{\lstinline[mathescape]~#1~}
\newcommand{\ciff}{\ \leftrightarrow\ }
\newcommand{\hyp}{\tac{H}}
\newcommand{\hypB}{\tac{G}}
\newcommand{\var}{\tac{x}}
\newcommand{\varB}{\tac{y}}

\newtheorem*{nlemma}{Lemma}


\title{Divisibility of multinomial coefficients}
\author{Jules Jacobs}

\begin{document}
\maketitle
\begin{abstract}
  Together with Ike Mulder we discovered a fun little divisibility property of multinomial coefficients (probably well-known!).
\end{abstract}

Our starting point is that not only the binomial coefficient $\frac{(a+b)!}{a! b!}$ is a whole number,
but also the Catalan numbers $\frac{(2n)!}{n! (n+1)!} = \frac{(2n)!}{n!n!} / (n+1)$ are whole numbers.
That $(a+b)!$ is divisible by $a! b!$ is already a small miracle, but the Catalan numbers show that it can be divided even further in certain cases. Our question is: does this generalize to other binomial coefficients?

The answer turns out to be yes: if $\gcd(a, b) = 1$ then $\frac {(a + b - 1)!}{a! b!}$ is a whole number.
Note that this implies the Catalan divisibility because $\gcd(n,n+1) = 1$.

The divisibility generalizes to multinomial coefficients:

\begin{nlemma}
  If $\gcd(a_1, \dots, a_n) = 1$ then
  \begin{align}
    \label{omultinom}
    \frac
      {(a_1 + \cdots + a_n - 1)!}
      {a_1! \cdots a_n!}
  \end{align}
  is a whole number.
\end{nlemma}
\begin{proof}
  From the $\gcd$ assumption, we have integers $k_1,\dots,k_n$ such that
  \begin{align*}
    1 = a_1 k_1 + \cdots + a_n k_n
  \end{align*}
  Multiply both sides by (\ref{omultinom}):
  \begin{align*}
    \frac
      {(a_1 + \cdots + a_n - 1)!}
      {a_1! \cdots a_n!} &=
        \frac
          {(a_1 + \cdots + a_n - 1)!}
          {a_1! \cdots a_n!} a_1 k_1
        + \cdots +
        \frac
          {(a_1 + \cdots + a_n - 1)!}
          {a_1! \cdots a_n!} a_n k_n \\
      &=
      \frac
          {(a_1 + \cdots + a_n - 1)!}
          {(a_1 - 1)! \cdots a_n!} k_1
        + \cdots +
        \frac
          {(a_1 + \cdots + a_n - 1)!}
          {a_1! \cdots (a_n - 1)!} k_n
  \end{align*}
  The right hand side is an integer because multinomial coefficients are integers.
\end{proof}

Slightly more generally, the proof shows that
  $\frac
    {(a_1 + \cdots + a_n - 1)! \gcd(a_1, \dots, a_n)}
    {a_1! \cdots a_n!}$
is always a whole number.
\end{document}
\documentclass[a4paper, 11pt]{article}
\usepackage[a4paper,left=2.5cm,right=2.5cm,top=2.5cm,bottom=2.5cm]{geometry}
\usepackage[utf8]{inputenc}
\usepackage[T1]{fontenc}
\usepackage[bitstream-charter]{mathdesign}
\usepackage{microtype}
\usepackage{textcomp}

\usepackage{hyperref}
\hypersetup{
    colorlinks=true,
    linkcolor=blue,
    filecolor=magenta,
    urlcolor=blue,
}
\usepackage{listings}
\usepackage{color}
\definecolor{lightgray}{rgb}{.9,.9,.9}
\definecolor{darkgray}{rgb}{.4,.4,.4}
\definecolor{purple}{rgb}{0.65, 0.12, 0.82}

\lstdefinelanguage{JavaScript}{
  keywords={typeof, new, true, false, catch, function, return, null, catch, switch, var, if, in, while, do, else, case, break, val, then},
  keywordstyle=\color{blue},
  ndkeywords={class, export, boolean, throw, implements, import, this},
  ndkeywordstyle=\color{darkgray}\bfseries,
  identifierstyle=\color{black},
  sensitive=false,
  comment=[l]{//},
  morecomment=[s]{/*}{*/},
  commentstyle=\color{purple}\ttfamily,
  stringstyle=\color{red}\ttfamily,
  morestring=[b]',
  morestring=[b]"
}

\lstset{
   language=JavaScript,
   extendedchars=true,
   basicstyle=\ttfamily,
   showstringspaces=false,
   showspaces=false,
}

\usepackage{amsmath}
\usepackage[shortlabels]{enumitem}

\newcommand{\N}{\mathbb{N}}
\newcommand{\Z}{\mathbb{Z}}
\newcommand{\Q}{\mathbb{Q}}
\newcommand{\R}{\mathbb{R}}
\newcommand{\sort}{\mathsf{sort}}


\title{Sorting Real Numbers, Constructively}
\author{Jules Jacobs}
\begin{document}
\maketitle

This note was inspired by a talk of Mikkel Kragh Mathiesen at OPLSS.

\section{Introduction}

We can represent a real number $r\in \R$ as Cauchy sequence $r_i \in \Q$ of rational numbers that converges to the real number we want to describe. We can represent an infinite sequence $r_i$ as a computer program that calculates the function $r : \N \to \Q$, with which we can get arbitrarily precise approximations to $r$.

In order get a guaranteed $\epsilon$-approximation, we must also have a constructive witness to the Cauchy property, that is, a way to determine how far into the sequence we have to look: given an $\epsilon > 0$ we must be able to determine an $n \in \N$ such that $r_{n+1},r_{n+2},\dots \in r_n \pm \epsilon$, where $r_n \pm \epsilon = \{x : r_n - \epsilon \leq x \leq r_n + \epsilon \}$ is an interval with radius $\epsilon$ around $r_n$. One can do many operations on these real numbers:
\begin{itemize}
  \item Arithmetic with $(r+s)_i = r_i + s_i$.
  \item Calculate $x = \sqrt{2}$ with $x_0 = 1$ and $x_{n+1} = \frac{x_n + 2/x_n}{2}$.
  \item Calculate transcendental functions $y = \sin(x)$ by taking $y_n = f_n(x_n)$ where $f_n$ is the $n$-term Taylor series of $\sin$.
\end{itemize}

\section{What Constructivists Cannot Do}

One thing you cannot do constructively is define a function that compares real numbers for equality. There are many different representations of the number zero: $x_n = 0$ and $y_n = 1/n$ because they both converge to zero. We say that two numbers $x,y$ are equal if their difference $x-y$ is zero. It is not possible to constructively determine whether two numbers are equal, because you'd have to look infinitely far into the sequence. For the same reason, it's not possible to determine whether $x < y$ or $x \leq y$ for arbitrary real numbers $x,y$. This remains true even if we had access to the source code of the computer programs for calculating the sequences $x,y$, due to Gödel's incompleteness.

It would therefore seem to be impossible to sort an array of real numbers $[x_1,x_2,\dots,x_n]$. The way we usually sort arrays is to determine a permutation that makes the array sorted, but in order to determine that permutation we'd have to compare the $x_i$ with each other.

In general, it is only possible to calculate \emph{continuous} functions of real numbers. The function $f(x) = 1$ if $x < 2$ and $f(x) = 0$ otherwise is not continuous: changing $x$ by a little bit may change $f(x)$ by a discrete step. The same holds for the permutation: changing one of the $x_i$ by a little bit may change the permutation required to put the array in sorted order, which is a discrete step change.

The situation is even worse. We usually say that a sort function is correct if its output is sorted, and if the output is a permutation of the input. Constructively, the latter entails a method to determine that permutation, which is impossible.

\section{A Glimmer of Hope}

Let $\mathsf{sort} : \R^n \to \R^n$ be a sort function, presumably given to us by a non-constructivist. The $\sort$ function itself \emph{is} continuous! To intuitively see this, let $[\dots, a, b, \dots]$ be a sorted array where each element is strictly smaller than the rest. If we slowly increase the value of $a$ while keeping the other values fixed, there will be some point when $a > b$ and the sorted array becomes $[\dots, b, a, \dots]$. This may look like a non-continuous change, but it's not. To see this, let us consider what that change looks like:
\begin{align*}
  [\dots, 1.7, 2.0, \dots] \\
  [\dots, 1.8, 2.0, \dots] \\
  [\dots, 1.9, 2.0, \dots] \\
  [\dots, 2.0, 2.0, \dots] \\
  [\dots, 2.0, 2.1, \dots] \\
  [\dots, 2.0, 2.2, \dots] \\
  [\dots, 2.0, 2.3, \dots] \\
\end{align*}

It looks like we first increased $a$, and when $a$ became equal to $b$, we started increasing $b$. This is a continuous change: a gradual change in the input resulted in a gradual change in the output. We did switch \emph{which} entry we were increasing, but two adjacent arrays in the preceding list only differ by a small amount. We therefore have some hope: considerations of continuity do not rule out a constructive $\sort$ function.

\section{Sorting Reals}

Similar to the preceding section, we can view an array $[a, b,\dots,z]$ of real numbers as a matrix of rational numbers, where the $i$-th row is an array of the $i$-th elements of the Cauchy sequences:

\[
\begin{matrix}
  [a_0, & b_0, & \dots, & z_0] \\
  [a_1, & b_1, & \dots, & z_1] \\
  [a_2, & b_1, & \dots, & z_2] \\
  \vdots & \vdots & \cdots & \vdots \\
\end{matrix}
\]

The output of our sorting algorithm must again be such a matrix. Since each row is an array of rational numbers, we can simply sort each row:

\[
\begin{matrix}
  \sort &[a_0, & b_0, & \dots, & z_0] \\
  \sort &[a_1, & b_1, & \dots, & z_1] \\
  \sort &[a_2, & b_1, & \dots, & z_2] \\
  \vdots&  \vdots & \vdots & \cdots & \vdots \\
\end{matrix}
\]

Each column of the sorted matrix is a sequence of rational numbers, so it is potentially a real number. To prove that this method is correct, we must show:
\begin{enumerate}
  \item Each column is a Cauchy sequence
  \item The output real numbers are sorted
  \item The output real numbers are a permutation of the input
\end{enumerate}

\noindent The second property clearly holds: if $a_i \leq b_i$ for all $i$, then also $a \leq b$ in the sense of real numbers. I hope that the first property seems plausible after the preceding section, but the third property may seem obviously false: obviously the columns of the sorted matrix are not a permutation of the original columns.

This is true, but remember that equality of real numbers $a,b$ is \emph{not} equality of $a_i = b_i$ for all $i$. Since we have already seen that we cannot constructively establish property 3, we reason classically.

First, let us suppose that the real numbers $a,b,\dots,z$ are all distinct. This means that there is some index $i$ beyond which the Cauchy sequences of $a,b,\dots,z$ all separate: the relative order of the rational numbers in their Cauchy sequence no longer changes. The permutation that $\sort$ will apply stabilizes after row $i$. Since only the tail of the sequences matters, the output of $\sort$ will be a permutation of the input in this case.

The interesting case is when some of the real numbers are equal. If $a = b$ then the relative order of the rationals in their Cauchy sequence may never stabilize. Consider $a_i = 0$ and $b_i = (-1)^i / i$. At even $i$ we have $a_i < b_i$ but at odd $i$ we have $b_i < a_i$. Thus, the permutation that $\sort$ applies will never stabilize.

If we group the input array in groups of real numbers that are equal, then the relative order of the groups \emph{will} stabilize because their Cauchy sequences will separate, but the order within each group may not. The key point is that \emph{this does not matter}: since the real numbers in each group are all equal, shuffling their Cauchy sequences around doesn't change a thing. If we have two real numbers $a=b$, then we can create a new real number $c$ by picking $c_i$ to be $a_i$ or $b_i$ arbitrarily, and we'll still have $a=b=c$.

The conclusion is that we can define the $\sort$ function constructively, but proving that its output is a permutation of the input can only be done classically.

\end{document}
\documentclass[a4paper, 11pt]{article}
\usepackage[a4paper,left=2.5cm,right=2.5cm,top=2.5cm,bottom=2.5cm]{geometry}
\usepackage[utf8]{inputenc}
\usepackage[T1]{fontenc}
\usepackage[bitstream-charter]{mathdesign}
\usepackage{microtype}
\usepackage{textcomp}

\usepackage{mathtools}
\newcommand\SetSymbol[1][]{\nonscript\:#1\vert\allowbreak\nonscript\:\mathopen{}}
\providecommand\given{} % to make it exist
\DeclarePairedDelimiterX\Set[1]\{\}{\renewcommand\given{\SetSymbol[\delimsize]}#1}

\usepackage{hyperref}
\hypersetup{
    colorlinks=true,
    linkcolor=blue,
    filecolor=magenta,
    urlcolor=blue,
}
\usepackage{listings}
\usepackage{color}
\definecolor{lightgray}{rgb}{.9,.9,.9}
\definecolor{darkgray}{rgb}{.4,.4,.4}
\definecolor{purple}{rgb}{0.65, 0.12, 0.82}

\lstdefinelanguage{JavaScript}{
  keywords={typeof, new, true, false, catch, function, return, null, catch, switch, var, in, while, do, else, case, break, val, then, Inductive, Fixpoint, match, end},
  keywordstyle=\color{blue},
  ndkeywords={class, export, boolean, throw, implements, import, this},
  ndkeywordstyle=\color{darkgray}\bfseries,
  identifierstyle=\color{black},
  sensitive=false,
  comment=[l]{//},
  morecomment=[s]{/*}{*/},
  commentstyle=\color{purple}\ttfamily,
  stringstyle=\color{red}\ttfamily
}

\lstset{
   language=JavaScript,
   extendedchars=true,
   basicstyle=\small\ttfamily,
   showstringspaces=false,
   showspaces=false,
   literate=
    {forall}{$\forall$}1
    {->}{$\to$}1
    {=>}{$\implies$}1
    {fun}{$\lambda$}1
}

\newcommand{\N}{\mathbb{N}}
\newcommand{\Z}{\mathbb{Z}}
\newcommand{\Q}{\mathbb{Q}}
\newcommand{\R}{\mathbb{R}}

\usepackage{amsmath}
\usepackage{amsthm}

\usepackage[shortlabels]{enumitem}
\usepackage{mathpartir}
\newcommand{\mif}{\mathsf{if}\ }
\newcommand{\mthen}{\ \mathsf{then}\ }
\newcommand{\melse}{\ \mathsf{else}\ }
\newtheorem{theorem}{Theorem}[section]
\newtheorem{lemma}[theorem]{Lemma}
\theoremstyle{definition}
\newtheorem{definition}{Definition}[section]
\newtheorem{corollary}{Corollary}[theorem]


\title{A simple proof of the matrix-tree theorem, \\ upward routes, and the matrix-tree-cycle theorem}
\author{Jules Jacobs}
\begin{document}
\maketitle

\begin{abstract}
  The matrix-tree theorem states that the number of spanning trees in a graph is $\det L'$, where $L'$ is the Laplacian matrix $L$ of the graph, with any row and column deleted. We give a direct proof of the fact that $\det(xI + L)$ is the generating function of spanning forests with $k$ roots. Our proof does not rely on the Cauchy-Binet formula, yet is arguably simpler than the standard proof, and applies to weighted \& directed graphs as well.

  The proof is based on a lemma that if $A$ is the adjacency matrix of a graph where each vertex has at most one outgoing edge, then $\det(I - A) = 1$ if $A$ is a forest and $\det(I - A) = 0$ otherwise. We generalize this lemma to any graph, in which case $\det(I-xA)^{-1}$ is shown to be the generating function of \emph{upward routes}.

  Lastly, we generalize the matrix-tree theorem to a theorem about $\det(A + L)$ where $A$ is the adjacency matrix of a second graph, which reduces to counting spanning forests when $A = xI$ and to the fact that $\det(A)$ counts signed cycle covers when $L = 0$. The all-minors matrix-tree theorem follows as a corollary. For instance, the fact that $\det(L')$ counts spanning forests, where $L'$ is $L$ with any column $i$ and row $j$ deleted, follows by picking $A$ to have $A_{ij} = 1$ and zero elsewhere.
\end{abstract}

\section{Introduction}

Given finite sets of numbers $S_1, \dots, S_n \subset \R$, we have the identity:
\begin{align*}
  \left( \sum_{x_1 \in S_1} x_1 \right) \cdots \left( \sum_{x_n \in S_n} x_n \right) =
  \sum_{x_1 \in S_1} \cdots \sum_{x_n \in S_n} x_1 \cdots x_n
\end{align*}
On the right hand side, we get one term for every way of choosing $(x_1,\dots,x_n)\in S_1 \times \cdots \times S_n$. Similarly, given finite sets of vectors $S_1, \dots, S_n \subset \R^k$, we have the following identity, by multilinearity of the determinant:
\begin{align*}
  \det\left(\sum_{v_1 \in S_1} v_1\ \bigg\rvert\ \cdots\ \bigg\rvert\ \sum_{v_n \in S_n} v_n\right) =
  \sum_{v_1 \in S_1} \cdots \sum_{v_n \in S_n} \det \left(v_1 | \cdots | v_n\right)
\end{align*}
Where $\det(v_1 | \dots | v_n)$ is the determinant of a matrix with those columns. We shall see that Kirchoff's theorem can be obtained from this identity, but we first need some definitions and a lemma.

\begin{definition}
We define the concepts \emph{1-graph}, \emph{01-graph}, \emph{forest}, \emph{root}, and \emph{tree}:
 \begin{itemize}
  \item A \emph{1-graph} is a directed graph where each vertex has 1 outgoing edge.
  \item A \emph{01-graph} is a directed graph where each vertex has 0 or 1 outgoing edges.
  \item A \emph{forest} is a 01-graph with no cycles.
  \item The \emph{roots} are vertices with 0 outgoing edges.
  \item A \emph{tree} is a forest with one root.
 \end{itemize}
\end{definition}

If we start at a vertex $v$ in a 01-graph and keep following the unique outgoing edge, then we either loop in a cycle, or we end up at a vertex with no outgoing edges. Therefore a general 01-graph consists of roots and cycles, plus trees converging onto the roots and cycles. A 1-graph is a 01-graph with no roots, so regardless of where we start, we always end up in a cycle.

\section{The matrix-tree theorem}

We need a lemma that give us an indicator function for forests.

\begin{lemma}
  Let $A_G$ be the adjacency matrix of a 1-graph $G$, then
  \[
    \det(I - A_G) = \begin{cases}
      1 & \text{if $G$ is empty} \\
      0 & \text{if $G$ is not empty}
    \end{cases}
  \]
\end{lemma}
\begin{proof}
  If $G$ is empty, we have a $0\times 0$ matrix, which has determinant 1.
  If $G$ is not empty, then each column of $I - A_G$ has one $+1$ diagonal entry and one $-1$ entry from $A_G$, so the sum of the rows is zero, so $\det(I - A_G) = 0$. This remains true in the presence of self loops.
\end{proof}

\begin{lemma}
  Let $A_G$ be the adjacency matrix of a 01-graph $G$, then
  \[
    \det(I-A_G) = \begin{cases}
      1 & \text{if $G$ is a forest}\\
      0 & \text{if $G$ has a cycle}
    \end{cases}
  \]
\end{lemma}
\begin{proof}
We calculate the determinant by repeatedly performing Laplace expansion on a column $i$ that corresponds to a root. The column of a root has a single $+1$ entry on the diagonal, so performing Laplace expansion along this column deletes column $i$ and row $i$. Row $i$ contains all the incoming edges of the root. Therefore, this operation corresponds to deleting root $i$ and all its incoming edges from the graph. Deleting a root may create new roots. Repeating this process of deleting roots, the remaining 1-graph will be empty iff the original graph was a forest. Applying the previous lemma gives the desired result.
\end{proof}

An alternative shorter proof using eigenvalues:

\begin{proof}
  If $G$ is a forest then $A_G$ is nilpotent, so all its eigenvalues are $0$, so all the eigenvalues of $I - A_G$ are 1, so $\det(I - A_G) = 1$.
  If $G$ has a cycle, then $A_G$ has 1 as an eigenvalue (take the eigenvector that is 1 on the cycle and 0 elsewhere), so $I - A_G$ has 0 as an eigenvalue, so $\det(I - A_G) = 0$.
\end{proof}

\textbf{We now fix $n\times n$ matrices $A$ and $D$ over some ring:}

\begin{itemize}
  \item An arbitrary matrix $A$ of edge weights (with $A_{ij}$ being the weight of edge $i \to j$)
  \item A diagonal matrix $D$ of vertex weights (with $D_{ii}$ being the weight of vertex $i$).
\end{itemize}

\begin{definition}
  The Laplacian matrix $L$ is defined as having columns
  \[ L_i = \sum_j A_{ij}(e_i - e_j) \]
\end{definition}

\begin{definition}
  The weight of a forest $G$ is \[w(G) = \prod_{i \in \mathsf{roots}(G)} D_{ii} \prod_{(i \to j) \in \mathsf{edges}(G)} A_{ij} \]
\end{definition}

\medskip
We're ready to state Kirchoff's theorem for multiple-root forests in weighted directed graphs.

\begin{theorem} (Kirchoff, Tutte) The determinant $\det(D + L)$ is the weight-sum of all forests on $n$-vertices:
  \[
    \det(D + L) = \sum_{\text{forest}\ G} w(G)
  \]
\end{theorem}
\begin{proof}
  The $i$-th column of the matrix is $(D + L)_i = D_{ii} e_i + \sum_j A_{ij}(e_i - e_j)$, so \\
  \begin{align*}
    \hspace*{-1cm}
    \det(D + L) = \det \left(
    \begin{array}{c|c|c|c}
      D_{11} e_1         & D_{22} e_2         &        & D_{nn} e_n          \\
      +                  & +                  &        & +                   \\
      A_{11} (e_1 - e_1) & A_{21} (e_2 - e_1) &        & A_{n1} (e_n - e_1)  \\
      +                  & +                  & \cdots & +                   \\
      A_{12} (e_1 - e_2) & A_{22} (e_2 - e_2) &        & A_{n2} (e_n - e_2)  \\
      \vdots             & \vdots             &        & \vdots              \\
      A_{1n} (e_1 - e_n) & A_{2n} (e_2 - e_n) &        & A_{nn} (e_n - e_n)
    \end{array} \right)
    = \sum_{\text{01-graph}\ G} w(G) \det(I - A_G)
    = \sum_{\text{forest}\ G} w(G)
  \end{align*} \\
  The first step is by the definition of $L$. In the middle step we expand the determinant by multilinearity: we sum over all possible ways to pick one term of each column. To choose a 01-graph on $n$ vertices, is to choose for each vertex $i$ (column $i$) whether to make $i$ a root (term $D_{ii} e_i$) or to give $i$ an outgoing edge $i \to j$ (term $A_{ij}(e_i - e_j)$). Then we take the weights $D_{ii}$ and $A_{ij}$ out of the determinant, and we're left with $w(G)\det(I - A_G)$, where $A_G$ is the adjacency matrix of the chosen 01-graph. The final step is applying the lemma.
\end{proof}

\section{Upward routes}

We now know that when $A$ is the adjacency matrix of a 01-graph, then $\det(I - A) = 1$ if $G$ is a forest and $\det(I - A) = 0$ if $G$ has a cycle. One naturally wonders about the value of $\det(I - A)$ when $A$ is an arbitrary adjacency matrix. The goal of this section is a combinatorial interpretation of $\det(I - xA)$ and $\det(I - xA)^{-1}$ for an arbitrary adjacency matrix, which generalizes the lemma to arbitrary graphs.

\begin{definition}
  Given a directed graph $G$ with an order on the vertices, we define \emph{(strictly) upward loops} and \emph{(strictly) upward routes}:
  \begin{itemize}
    \item An \emph{upward loop} at vertex $i$ is a path from $i$ to $i$ that does not visit vertices lower than $i$.
    \item A \emph{strictly upward loop} at vertex $i$ is a path from $i$ to $i$ that only visits vertices higher than $i$ (except at the start/endpoint of the path, where it does visit $i$ itself).
    \item A \emph{(strictly) upward route} is a choice of (strictly) upward loop at each vertex.
  \end{itemize}
  Let $f_i(x)$ be the generating function of strictly upward loops of length $k$ at vertex $i$. Then \[ f^{*}_i(x) = (1 - f_i(x))^{-1} \] is the generating function of upward loops of length $k$ at vertex $i$, because an upward loop splits uniquely into a sequence of strictly upward loops. Furthermore, the generating functions $f(x)$ and $f^{*}(x)$ of (strictly) upward routes of $k$ edges are given by:
  \begin{align*}
    f(x) &= \prod_{i=1}^n f_i(x) &
    f^{*}(x) &= \prod_{i=1}^n f^{*}_i(x)
  \end{align*}
\end{definition}

Recall Cramer's rule:
\begin{theorem} (Cramer's rule)
  Let $A$ be an invertible matrix and let $A_{[i,j]}$ be the same with column $i$ and row $j$ deleted, then:
  \[ A^{-1}_{ij} = \frac{\det(A_{[i,j]})}{\det(A)} \]
\end{theorem}

From this, we get the following lemma that allows us to calculate $\det(A)^{-1}$ in terms of entries of inverses of submatrices of $A$:

\begin{lemma} Given a matrix $A$, provided both sides are defined,
  \[ \det(A)^{-1} = \prod_{i=0}^{n-1} (A_{[1\dots i, 1\dots i]})^{-1}_{11} \]
Where $A_{[1\dots i, 1\dots i]}$ is the matrix $A$ with the first $i$ rows and columns deleted.
\end{lemma}
\begin{proof}
  Cramer's rule implies:
\[
  \det(A)^{-1} = A^{-1}_{1,1} \cdot \det(A_{[1,1]})^{-1}
\]

Continuing this by induction, we get:
\begin{align*}
  \det(A)^{-1} &= A^{-1}_{1,1} \cdot \det(A_{[1,1]})^{-1} \\
    &= A^{-1}_{1,1} \cdot (A_{[1,1]})^{-1}_{11} \cdot \det(A_{[1..2,1..2]})^{-1} \\
    &= \cdots \\
    &= A^{-1}_{1,1} \cdot (A_{[1,1]})^{-1}_{11} \cdot (A_{[1..2,1..2]})^{-1}_{1,1} \cdots
        (A_{[1..n-1,1..n-1]})^{-1}_{1,1} \cdot 1
\end{align*}
\end{proof}

\begin{lemma}
  The generating function of upward loops is $((I - xA)_{[1\dots i-1, 1\dots i-1]})^{-1}_{11}$.
\end{lemma}
\begin{proof}
  Given an adjacency matrix $B$, $(I - xB)^{-1} = I + xB + x^2 B^2 + \cdots$ is the matrix generating function of paths, so $(I - xB)^{-1}_{11}$ is the generating function of loops from vertex 1 to 1. To obtain upward loops at vertex $i$ in $A$ we take $B = (I - xA)_{[1\dots i-1, 1\dots i-1]}$ with the first $i-1$ rows and columns deleted.
\end{proof}

We combine these lemmas to obtain:

\begin{theorem}
  The generating function of upward routes with $k$ edges is $\det(I - xA)^{-1}$.
\end{theorem}
\begin{proof}
  Combine the preceding two lemmas with $f^{*}(x) = \prod_{i=1}^n f^{*}_i(x)$.
\end{proof}

\begin{corollary}
  The number of upward routes of $k$ edges does not depend on the order of the vertices.
\end{corollary}
\begin{proof}
  If we permute the order of the vertices by a permutation $P$, the generating function
  \[
    \det(I - xPAP^{-1}) = \det(P(I - xA)P^{-1}) = \det(I - xA)
  \] stays the same. A bijective proof is left as an exercise.
\end{proof}

\begin{corollary} For an arbitrary adjacency matrix $A$,
  \[ \det(I - xA) = \prod_{i = 1}^n (1 - f_i(x)) \]
  Where $f_i(x)$ is the generating function of strictly upward loops at vertex $i$.
\end{corollary}
\begin{proof}
  Use the relationship between the generating functions:
  \[
    \det(I - xA) = \bar{f}(x)^{-1} = \prod_{i=1}^n \bar{f}_i(x)^{-1} = \prod_{i=1}^n (1 - f_i(x))
  \]
\end{proof}

Note that $\det(I - xA)$ is a polynomial, even though $f_i(x)$ is a power series. The main lemma follows as a corollary, which gives us a third proof of the lemma:

\begin{corollary}
  Let $A_G$ be the adjacency matrix of a 01-graph $G$, then
  \[
    \det(I-A_G) = \begin{cases}
      1 & \text{if $G$ is a forest}\\
      0 & \text{if $G$ has a cycle}
    \end{cases}
  \]
\end{corollary}
\begin{proof}
  If $G$ is a forest, then it has no strictly upward loops, so $f_i(x) = 0$, so $\det(I - xA_G) = 1$. If $G$ has a cycle, let $i$ be the lowest vertex on the cycle. Then $f_i(x) = x^k$ where $k$ is the length of the cycle. Now substitute $x = 1$ to obtain $\det(I - A_G) = 0$.
\end{proof}

\section{The matrix-tree-cycle theorem}

We shall now generalize the matrix-tree theorem about $\det(D + L)$ to the \emph{matrix-tree-cycle theorem} by allowing $D$ to be a general matrix. This theorem will express $\det(D + L)$ as a sum over 1-graphs with its edges labeled with T (for tree) or C (for cycle).

\begin{definition}
  A TC-labeled 1-graph is called \emph{tree-cyclic} if the T-edges form a forest, and each C-edge is part of a cycle.
\end{definition}

\begin{definition}
  The sign of a TC-labeled 1-graph $G$ is
  \[
   (-1)^{G} = (-1)^{\text{\#cycles} + \text{\#C-edges}}
  \]
  where \#cycles is the number of cycles of $G$ and \#C-edges is the total number of C-labeled edges in $G$.
\end{definition}

We associate a matrix $M_G$ with a TC-labeled 1-graph $G$. This matrix will play the role that $I - A_G$ played in the matrix-tree theorem.

\begin{definition}
  The matrix $M_G$ is given by columns:
  \[
    (M_G)_i = \begin{cases}
      e_j & \text{if the outgoing edge $i \to j$ has label C} \\
      e_i - e_j & \text{if the outgoing edge $i \to j$ has label T} \\
    \end{cases}
  \]
\end{definition}

The following lemma generalizes the main lemma of the matrix-tree theorem.

\begin{lemma}
  The determinant of $M_G$ is given by:
  \[
    \det(M_G) = \begin{cases}
      (-1)^G & \text{if $G$ is tree-cyclic} \\
      0 & \text{otherwise}
    \end{cases}
  \]
\end{lemma}
\begin{proof}
  Calculate the determinant in steps:
  \begin{enumerate}
    \item Perform Laplace expansion on vertices with no predecessors. This makes the determinant zero if the successor is a C-edge, and deletes the corresponding vertex if the successor is a T-edge.
    \item We are now left with a disjoint set of cycles. If there is a cycle consisting solely of T-edges, the determinant is zero because the corresponding rows sum to zero.
    \item We are now left with a disjoint set of cycles where each cycle has at least one C-edge. Use the C-edges to turn the T-edges into C-edges by row operations. The determinant obtains obtains a $-1$ sign for each such switch.
    \item We are now left with a disjoint set of cycles where each cycle consists solely of C-edges. In other words, a permutation matrix. The determinant of a permutation matrix is $(-1)^{\text{\#cycles} + \text{\#edges}}$
  \end{enumerate}
  If there was a C-edge not part of a cycle we have obtained $0$ in step 1, and if there was a cycle among T-edges we have obtained $0$ in step 2. For the remaining graphs, we have obtained a $-1$ sign for each T-edge in the cycles, so together with $(-1)^{\text{\#cycles} + \text{\#edges}}$ we are left with $(-1)^{\text{\#cycles} + \text{\#C-edges}} = (-1)^G$.
\end{proof}

\textbf{We now fix $n\times n$ matrices $A$ and $D$ over some ring:}

\begin{itemize}
  \item An arbitrary matrix $A$ of weights for T-edges (with $A_{ij}$ being the weight of edge $i \xrightarrow{T} j$)
  \item An arbitrary matrix $D$ of weights for C-edges (with $D_{ij}$ being the weight of edge $i \xrightarrow{C} j$).
  \item As before, the Laplacian $L$ is given by $L_i = \sum_j A_{ij}(e_i - e_j)$.
\end{itemize}


\begin{definition}
  The weight of a tree-cyclic 1-graph $G$ is
  \[
    w(G) = \prod_{\text{C-edge}\ (i \to j) \in G} D_{ij} \prod_{\text{T-edge}\ (i \to j) \in G} A_{ij}
  \]
\end{definition}

We’re ready to state the matrix-tree-cycle theorem.

\begin{theorem} The determinant $\det(D+L)$ is the signed weight-sum of tree-cyclic graphs:
  \[
    \det(D + L) = \sum_{\text{tree-cyclic}\ G} (-1)^G w(G)
  \]
\end{theorem}
\begin{proof}
  The $i$-th column of the matrix is $(D + L)_i = \sum_j D_{ij} e_j + \sum_j A_{ij}(e_i - e_j)$, so \\
  \begin{align*}
    \hspace*{-1.5cm}
    \det(D + L) = \det \left(
    \begin{array}{c|c|c|c}
      D_{11} e_1         & D_{21} e_1         &        & D_{n1} e_1          \\
      +                  & +                  &        & +                   \\
      % D_{12} e_2         & D_{22} e_2         &        & D_{n2} e_2          \\
      \vdots             & \vdots             &        & \vdots              \\
      +                  & +                  &        & +                   \\
      D_{1n} e_n         & D_{2n} e_n         &        & D_{nn} e_n          \\
      +                  & +                  & \cdots & +                   \\
      A_{11} (e_1 - e_1) & A_{21} (e_2 - e_1) &        & A_{n1} (e_n - e_1)  \\
      +                  & +                  &        & +                   \\
      % A_{12} (e_1 - e_2) & A_{22} (e_2 - e_2) &        & A_{n2} (e_n - e_2)  \\
      \vdots             & \vdots             &        & \vdots              \\
      +                  & +                  &        & +                   \\
      A_{1n} (e_1 - e_n) & A_{2n} (e_2 - e_n) &        & A_{nn} (e_n - e_n)
    \end{array} \right)
    = \sum_{\text{TC-1-graph}\ G} w(G) \det(M_G)
    = \sum_{\text{tree-cyclic}\ G} (-1)^G w(G)
  \end{align*} \\
  In the middle step we have again expanded the determinant by multilinearity: we sum over all possible ways to pick one term of each column. To choose a TC-labeled 1-graph on $n$ vertices, is to choose for each vertex $i$ (column $i$) an outgoing edge $i \to j$ and a label C (term $D_{ij} e_j$) or T (term $A_{ij}(e_i - e_j)$) for this edge. The final step is applying the lemma.
\end{proof}

The matrix-tree theorem is recovered by taking $D$ to be diagonal. In that case, the number of cycles and the number of C-edges is equal, so $(-1)^{\text{\#cycles} + \text{\#C-edges}} = 1$ and there are no signs involved. Each C-labeled self-loop corresponds to a root.

If we take $L=0$ we get the fact that $\det(D)$ is the sum of signed cycle covers. A cycle cover of $n$ vertices is a choice of edges such that each vertex has precisely one incoming and one outgoing edge, and its sign is the sign of the permutation that this graph depicts.

By taking $D_{ij} = 1$ for some $i,j$ and zero elsewhere, we also ensure that the sign $(-1)^G = 1$, because in this case we have one C-edge and one cycle. Thus, $\det(D+L)$ will be the number of spanning trees rooted at $i$. The choice of $j$ does not matter. By developing the determinant and using $\det(L) = 0$ we see that this is equal to $\det(L')$ where $L'$ is obtained from $L$ by deleting row $i$ and column $j$.

By taking more off-diagonal entries of $D$ to be nonzero, and by taking the corresponding rows and columns to be zero in $A$ (and thus in $L$), we obtain the all-minors matrix-tree theorem.

\end{document}